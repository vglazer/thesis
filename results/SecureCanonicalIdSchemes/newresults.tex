\documentclass[12pt,twoside]{article}

% Standard packages
% standard packages
\usepackage{enumerate,psfrag,amsfonts,amssymb,amsmath,amsthm,url,epic,eepic}
\ifx\pdfoutput\undefined
  \usepackage[dvips]{graphicx}
\else
  \usepackage[pdftex]{graphicx}
\fi


%% Define a new 'leo' style for the package that will use a smaller font.
\makeatletter
\def\url@smallstyle{\@ifundefined{selectfont}{\def\UrlFont{\sf}}{\def\UrlFont{\small\ttfamily}}\Url@do
}
\makeatother

%% Now actually use the newly defined style.
\urlstyle{small}

\usepackage{nopageno}

% Narrow margins
\setlength{\textwidth}{7in}
\setlength{\topmargin}{-0.575in}
\setlength{\textheight}{9.25in}
\setlength{\oddsidemargin}{-.25in}
\setlength{\evensidemargin}{-.25in}



% Macros
%*****************
% Important Sets *
%*****************

% The naturals
\newcommand{\nats}{\mathbb{N}}

% The integers
\newcommand{\ints}{\mathbb{Z}}

% The positive positive integers
\newcommand{\pints}{\mathbb{Z^+}}

% The reals
\newcommand{\reals}{\mathbb{R}}

% The interval [0,1]
\newcommand{\zone}{[0,1]}

% Binary strings
\newcommand{\strs}[1]{\{0,1\}^{#1}}


%*****************
% Script Letters *
%*****************

% script B
\newcommand{\fanb}{\mathcal{B}}

% script C
\newcommand{\fanc}{\mathcal{C}}

% script D
\newcommand{\fand}{\mathcal{D}}

% script E
\newcommand{\fane}{\mathcal{E}}

% script F
\newcommand{\fanf}{\mathcal{F}}

% script G
\newcommand{\fang}{\mathcal{G}}

% script H
\newcommand{\fanh}{\mathcal{H}}

% script M
\newcommand{\fanm}{\mathcal{M}}

% script O
\newcommand{\fano}{\mathcal{O}}

% script P
\newcommand{\fanp}{\mathcal{P}}

% script R
\newcommand{\fanr}{\mathcal{R}}

% script S 
\newcommand{\fans}{\mathcal{S}}

% script T 
\newcommand{\fant}{\mathcal{T}}

% script U 
\newcommand{\fanu}{\mathcal{U}}

% script Z
\newcommand{\fanz}{\mathcal{Z}}

%*********************
% Special Constructs *
%*********************

% |x| over n
\newcommand{\overn}[1]{\frac{|#1|}{n}}

% ensembles and circuits
\newcommand{\ens}[2]{\{#1_#2\}_{#2 \in \nats}}

% probability
\newcommand{\pr}[3]{\text{Pr}_{#2 \in #3}[#1]}

% fiat-shamir signature scheme
\newcommand{\fs}[2]{SIG_{#2}(#1)}

% Merkle tree committment
\newcommand{\merk}[2]{TC_{f_#2}(#1)}

% ceiling
\newcommand{\ceil}[1]{\lceil #1 \rceil}

% floor
\newcommand{\floor}[1]{\lfloor #1 \rfloor}

% encoding into strings
\newcommand{\enc}[1]{\langle #1 \rangle}

% function definition
\newcommand{\func}[3]{#1 : #2 \to #3}



% Definitions
% Force proofs to end in a filled-in square rather than a hollow one.
\renewcommand{\qedsymbol}{$\blacksquare$}

% For dagger footnotes
\long\def\symbolfootnote[#1]#2{\begingroup%
\def\thefootnote{\fnsymbol{footnote}}\footnote[#1]{#2}\endgroup}

% When numbering sections, don't include the chapter number
\def\thesection{\arabic{section}} 

% A basic theorem environment
\theoremstyle{plain}
\newtheorem*{thm}{Theorem}

% A basic definition environment
\theoremstyle{definition}
\newtheorem*{defn}{Definition}

\theoremstyle{remark}
\newtheorem*{rem}{Remark}



\title{Secure canonical identification schemes yield Fiat-Shamir signature
schemes secure in the random oracle model}
\author{Victor Glazer}

\begin{document}
\maketitle
\section*{Background}
\begin{itemize}
\item An identification scheme $ID = (G,P,V)$ is said to be \emph{canonical} 
if it is a three-round, public-coin scheme. The prover $P$ goes first; his 
move is called the \emph{commitment}, denoted by \textsc{Cmt}. The verifier 
$V$ replies with a random \emph{challenge} \textsc{Ch}, consisting of his 
random bits. $P$ then sends a \emph{response} \textsc{Rsp} to $V$, who either 
accepts or rejects the transcript $(\textsc{Cmt},\textsc{Ch},\textsc{Rsp})$. 

\begin{rem}
To streamline the presentation, we assume throughout that $n \leq |pub| 
\leq |pri|$ and $|\textsc{Ch}| = n$, where $n$ is the security parameter.
\end{rem}

\item The \emph{Fiat-Shamir transform} takes a canonical identification scheme 
$ID = (G,P,V)$ and a hash function $h: \strs{*} \to \strs{n}$, and outputs the
following signature scheme $SIG_h(ID) = (GEN, SIGN, VER)$. 

\medskip\noindent
The key generation algorithm, $GEN$, is identical to $G$\footnote{Strictly 
speaking, the Fiat-Shamir transform should be defined with respect to an 
ensemble $\ens{\fanh}{n}$ rather than an individual function $h$. This is 
because, for any fixed $h$, it's easy to come up with (contrived) secure 
canonical id schemes which yield insecure Fiat-Shamir signature schemes. In 
this setting, $GEN$ randomly chooses a key $k \in \fanh_n$ and appends it 
to $pub$ to form the public key $PK$. The private key $SK$ is simply set to 
$pri$.}. 

\medskip\noindent
To sign a message $m \in \strs{*}$, $SIGN_{pri}$ obtains a commitment
\textsc{Cmt} by running $P$ on $pri$, computes $y = h(\textsc{Cmt},m)$ and 
gives $y$ to $P$ as the challenge \textsc{Ch}. $P$ responds with \textsc{Rsp}
(recall that the \emph{completeness} property of $ID$ ensures that $P$ can
correctly answer any challenge \textsc{Ch}). $SIGN_{pri}$ then outputs 
$\sigma = (\textsc{Cmt},\textsc{Rsp})$ as the signature of $m$. 

\medskip\noindent
To determine whether $\sigma = (\textsc{Cmt},\textsc{Rsp})$ 
is a legitimate signature of $m$, $VER_{pub}$ computes 
$y = h(\textsc{Cmt},m)$ and runs $V_{pub}$ on $(\textsc{Cmt},y,\textsc{Rsp})$. 

\item The random oracle model is a popular approach to analyzing the security
of cryptographic protocols involving hash functions. Let $h: \strs{*} \to
\strs{n}$ be a hash function and $\pi(h)$ be a protocol which utilizes $h$.
To prove that $\pi(h)$ is secure in 
the random oracle model, we proceed as follows. All parties --- including the 
adversary --- are equipped with a random oracle $\fanr : \strs{*} \to 
\strs{n}$, and evaluations of $h$ are replaced with queries to $\fanr$. The 
adversary's success probability, now also taken over the randomness of $\fanr$,
is then shown to be negligible in $n$ under plausible hardness assumptions.
\end{itemize}

\section*{Results}

\begin{thm}
Let $ID = (G,P,V)$ be a secure canonical identification scheme and $h:
\strs{*} \to \strs{n}$ be a hash function. Then the signature scheme 
$\fs{ID}{h} = (GEN,SIGN,VER)$ is secure in the random oracle model.
\end{thm}
\begin{proof}
Let $F$ be a forger that breaks the security of $\fs{ID}{h}$ in the random
oracle model. $F$ has access to a random oracle $\fanr : \strs{*} \to \strs{n}$
and a signature oracle $\fans$. 

\medskip\noindent
When queried on a string $s$ for the first time, $\fanr$ chooses $r \in
\strs{n}$ uniformly at random and sets $\fanr(s) = r$. Subsequently, $\fanr$
responds with $r$ whenever queried on $s$. To produce a signature 
$\sigma = (\textsc{Cmt},\textsc{Rsp})$ of message $m$, $\fans$ first obtains a 
commitment \textsc{Cmt} from $P_{pri}$ and then gives him 
a challenge $\fanr(\textsc{Cmt},m)$, to which $P_{pri}$ responds with 
\textsc{Rsp}. Observe that $\fanr$'s replies must be consistent with 
those of $\fans$. For instance, if $\fans(m) = 
(\textsc{Cmt},\textsc{Rsp})$ it should be the case that 
$V_{pub}(\textsc{Cmt},\fanr(\textsc{Cmt},m),\textsc{Rsp}) = 1$.

\medskip\noindent
Let $\varnothing \subset \fanc \subset \strs{*}$ be the space $P$ draws his 
commitments from. We make no additional assumptions about $\fanc$, so that 
$\fanc = \strs{n}$ and $\fanc = \{\lambda\}$ are equally legitimate choices. 
Also, denote the message whose signature $F$ tries to forge by $m^*$, and its 
supposed signature by $\sigma^* = (\textsc{Cmt}^*, \textsc{Rsp}^*)$.

\medskip\noindent
We make a few simplifying assumptions about $F$, insisting that 
he have the following ``normal form'': 
\begin{enumerate}[$(i)$]
\item $F$ never queries $\fanr$ on the same string more than once. 

\item All of $F$'s random oracle queries are of the form
$\fanr(\textsc{Cmt},m)$, where $\textsc{Cmt} \in \fanc$ and $m \in \strs{*}$.

\item $F$ queries $\fanr$ on $(\textsc{Cmt}^*,m^*)$ at some point. This 
special query is called the ``crucial query''. 
\end{enumerate}
It isn't too hard to show that if a successful forger 
exists, then there exists one satisfying the above three properties. 

\medskip\noindent
Suppose that $F$ fails to have property $(i)$, so that he queries $\fanr$ on 
some string $s$ multiple times. Let $F'$ be the same as $F$, except that $F'$ 
writes $ans = \fanr(s)$ down on an unused portion of his working tape the first
time $\fanr$ is queried on $s$, and all subsequent random oracle queries about 
$s$ are answered by looking $ans$ up. Since $\fanr$ is a function, $F'$'s 
success probability is unchanged, yet he only queries $\fanr$ on $s$ once. 
If there is another string $s'$ on which $F'$ queries $\fanr$ multiple times, 
we can repeat the above process to get a new forger $F''$ which queries $\fanr$
on $s'$ once. Proceeding in this fashion, we eventually obtain a forger 
who doesn't query $\fanr$ on any string more than once, and whose success 
probability is identical to that of $F$. This assumption guarantees that $F$
doesn't repeat random oracle queries, so that $\fanr$'s answers are always 
random.

\medskip\noindent
Now suppose that $F$ fails to have property $(ii)$, so that at least one of his 
random oracle queries is not of the form $\fanr(\textsc{Cmt},m)$. Let $F'$ be 
the same as $F$, except that all malformed $\fanr$ queries are answered 
randomly. Since $\fanr$'s replies are also random, these answers have exactly 
the right distribution. Notice that there is no interplay between answers to 
$\fans$ queries and malformed $\fanr$ queries, so no inconsistencies are 
introduced. The new forger's success probability is therefore identical to 
that of $F$, and all of his $\fanr$ queries are well-formed. 

\medskip\noindent
Finally, suppose that $F$ fails to have property $(iii)$, namely that he never 
queries $\fanr$ on $(\textsc{Cmt}^*,m^*)$. Let $F'$ be the same as $F$, except
that instead of outputting $(m^*,\sigma^*)$ right away, $F'$ first queries
$\fanr$ on $(\textsc{Cmt}^*,m^*)$. $F'$'s success probability is 
identical to that of $F$, since the extra $\fanr$ query does not affect his 
output. It is also worth noting that the new forger doesn't violate
assumptions $(i)$ and $(ii)$, because the extra $\fanr$ query is both
new and well-formed.

\medskip\noindent
We are now ready to describe an impersonator $I$ which breaks the security of 
$ID$. Recall that $I$'s goal is to get the verifier $V$ to 
accept by interacting with him in the role of the prover $P$. $I$ is allowed
to first interact with $P$ in the role of $V$ polynomially many times. Since 
$I$ is trying to break the \emph{active} security of $ID$, he can send $P$ 
whatever messages he likes.

\medskip\noindent
Consider the experiment where a pair of keys $(pub,pri)$ is generated by
running $G$ on $1^n$, and $I$ is given the public key $pub$.

\medskip\noindent
Let $q_\fanr(n)$ and $q_\fans(n)$ denote the number of times $F$ queries 
$\fanr$ and $\fans$, respectively, and set $q(n) = q_\fanr(n) + q_\fans(n)$. 
$I$ first interacts with $P_{pri}$ $q(n)\cdot q_\fans(n)$ times in order to 
construct ``transcript blocks'' $\fanb_1,\ldots,\fanb_{q(n)}$. Each 
block is made up of $q_\fans(n)$ transcripts of the form 
$(\textsc{Cmt},r,\textsc{Rsp})$, obtained as follows. $I$ first receives a 
commitment $\textsc{Cmt} \in \fanc$ from $P_{pri}$. Next, $I$ sends a 
challenge $r \in \strs{n}$ to $P_{pri}$. If \textsc{Cmt} does not 
appear in any of the transcripts added to the block so far, $r$ is chosen 
randomly. Otherwise, $r$ is set to the challenge associated with \textsc{Cmt}. 
$P_{pri}$ responds with \textsc{Rsp}, and the transcript 
$(\textsc{Cmt},r,\textsc{Rsp})$ is added to the block.

\medskip\noindent
$I$ next guesses the index of $F$'s ``crucial query'' by randomly choosing
$k \in \{1,\ldots,q_\fanr(n)\}$. 

\medskip\noindent
$I$ now begins to simulate $F$. Note that assumption $(ii)$ above
enables us to associate a unique message with every $\fanr$ query 
$F$ makes. Since $\fans$ queries explicitly reference a message, every oracle
query made by $F$ therefore has a message unambiguously associated with it. 

\medskip\noindent
Let $m_1,m_2,m_3,\ldots$ be the \emph{distinct} messages associated with $F$'s
oracle queries. There are at most $q(n)$ of these, since in the worst case 
every query concerns a different message. $I$ answers queries associated with 
$m_i$ using transcripts stored in block $\fanb_i$. The answer to the 
$j^{th}$ $S(m_i)$ query is $(\textsc{Cmt},\textsc{Rsp})$, 
where $(\textsc{Cmt},r,\textsc{Rsp})$ is the $j^{th}$ transcript in 
$\fanb_i$. If \textsc{Cmt} appears in any of the transcripts stored in 
$\fanb_i$, then the answer to $\fanr(\textsc{Cmt},m_i)$ is $r$, the 
challenge associated with \textsc{Cmt}. Otherwise, $\fanr(\textsc{Cmt},m_i)$ 
is answered randomly. 

\medskip\noindent
The $k^{th}$ random oracle query, $\fanr(\textsc{Cmt}',m')$, is handled 
specially. $I$ sends $\textsc{Cmt}'$ to $V_{pub}$, receives a challenge 
\textsc{Ch} in reply and gives \textsc{Ch} to $F$ as the answer to 
$\fanr(\textsc{Cmt}',m')$. Let $k^*$ denote
the \emph{true} index of the ``crucial query''. Observe that if  
$k \neq k^*$, then $I$'s simulation of $F$ may break down. What if 
$F$ requests to see some signatures of $m'$? One of these could well  
involve $\textsc{Cmt}'$.
In that case, the correct answer to $\fanr(\textsc{Cmt}',m')$ is the 
corresponding challenge, $r$, which almost certainly differs from \textsc{Ch}. 
On the other hand, if $k = k^*$ then $m' = m^*$ and $\textsc{Cmt}' =
\textsc{Cmt}^*$. $F$ won't query $\fans$ on $m^*$ since that is the message
whose signature he is trying to forge, and $I$'s simulation of $F$ is perfect.

\medskip\noindent
Eventually, $F$ outputs a message $m^*$ together with an alleged signature
$\sigma^* = (\textsc{Cmt}^*,\textsc{Rsp}^*)$ of $m^*$. $I$ then sends
$\textsc{Rsp}^*$ to $V_{pub}$, who either accepts or rejects the transcript
$(\textsc{Cmt}',\textsc{Ch},\textsc{Rsp}^*)$.

\medskip\noindent
Let $p_F(n)$ and $p_I(n)$ denote the success probabilities of $F$ and $I$,
respectively. Note that $q_\fanr(n) \leq q(n) \leq t_F(n) \leq n^c$ for some
$c$, where $t_F(n)$ is the running time of $F$. Also observe that 
$p_F(n) \geq \frac{1}{n^d}$ for some $d$ and infinitely many $n$, because $F$ 
breaks the security of $\fs{ID}{h}$ in the random oracle model.
%We have
%\begin{equation*}
%p_F(n) = \text{Pr}\left[
%\begin{aligned} 
%& (pub,pri) \gets GEN^\fanr(1^n); (m^*,\textsc{Cmt}^*,
%\textsc{Rsp}^*) \gets F^{\fanr,\fans}_{pub}; \\ 
%& VER^\fanr_{pub}(m^*,\textsc{Cmt}^*,\textsc{Rsp}^*) = 
%V_{pub}(\textsc{Cmt}^*,\fanr(\textsc{Cmt}^*,m^*),\textsc{Rsp}^*) = 1
%\end{aligned}
%\right], 
%\end{equation*}
%where the probability is taken over the coins of $GEN$ and $F$, as well as the
%randomness of $\fanr$ and $\fans$. We also have
%\begin{equation*}
%p_I(n) = \text{Pr}\left[
%\begin{aligned}
%(pub,pri) \gets G(& 1^n); \textsc{Ch} \gets \strs{n};
%(\textsc{Cmt}',\textsc{Rsp}^*) \gets I^{P_{pri}}_{pub};\\ 
%& V_{pub}(\textsc{Cmt}',\textsc{Ch},\textsc{Rsp}^*) = 1
%\end{aligned}
%\right],
%\end{equation*}
%where the probability is taken over the coins of $G$, $I$ and $P_{pri}$, as
%well as the choice of $\textsc{Ch}$.

\medskip\noindent 
Since $I$'s simulation of $F$ is perfect provided he correctly guesses the
index of the ``crucial query'', we have:
\begin{align*}
p_I(n) & = \text{Pr}[V_{pub}(\textsc{Cmt}',\textsc{Ch},\textsc{Rsp}^*) = 1] \\
& = \text{Pr}[V_{pub}(\textsc{Cmt}',\textsc{Ch},\textsc{Rsp}^*) = 1 
\wedge k' = k^*] + \text{Pr}[V_{pub}(\textsc{Cmt}',\textsc{Ch},\textsc{Rsp}^*)
= 1 \wedge k' \neq k^*] \\
& \geq \text{Pr}[V_{pub}(\textsc{Cmt}',\textsc{Ch},\textsc{Rsp}^*) = 1 
\wedge k' = k^*] \\
& = \text{Pr}[V_{pub}(\textsc{Cmt}',\textsc{Ch},\textsc{Rsp}^*) 
= 1 | k = k^*] \cdot \text{Pr}[k' = k^*] \\
& = p_F(n) 
\cdot \frac{1}{q_\fanr(n)} \geq \frac{p_F(n)}{n^c}
\geq \frac{1}{n^{c+d}} \text{ for infinitely many } n.
\end{align*}
This shows that $p_I(n)$ is non-negligible in $n$, so $I$ breaks the 
security of $ID$.
\end{proof}

\end{document}

