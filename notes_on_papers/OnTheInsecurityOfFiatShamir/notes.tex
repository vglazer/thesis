\documentclass[12pt,twoside]{article}

% Standard Packages
% standard packages
\usepackage{enumerate,psfrag,amsfonts,amssymb,amsmath,amsthm,url,epic,eepic}
\ifx\pdfoutput\undefined
  \usepackage[dvips]{graphicx}
\else
  \usepackage[pdftex]{graphicx}
\fi


%% Define a new 'leo' style for the package that will use a smaller font.
\makeatletter
\def\url@smallstyle{\@ifundefined{selectfont}{\def\UrlFont{\sf}}{\def\UrlFont{\small\ttfamily}}\Url@do
}
\makeatother

%% Now actually use the newly defined style.
\urlstyle{small}


% Narrow Margins
\setlength{\textwidth}{7in}
\setlength{\topmargin}{-0.575in}
\setlength{\textheight}{9.25in}
\setlength{\oddsidemargin}{-.25in}
\setlength{\evensidemargin}{-.25in}



% Macros
%*****************
% Important Sets *
%*****************

% The naturals
\newcommand{\nats}{\mathbb{N}}

% The integers
\newcommand{\ints}{\mathbb{Z}}

% The positive positive integers
\newcommand{\pints}{\mathbb{Z^+}}

% The reals
\newcommand{\reals}{\mathbb{R}}

% The interval [0,1]
\newcommand{\zone}{[0,1]}

% Binary strings
\newcommand{\strs}[1]{\{0,1\}^{#1}}


%*****************
% Script Letters *
%*****************

% script B
\newcommand{\fanb}{\mathcal{B}}

% script C
\newcommand{\fanc}{\mathcal{C}}

% script D
\newcommand{\fand}{\mathcal{D}}

% script E
\newcommand{\fane}{\mathcal{E}}

% script F
\newcommand{\fanf}{\mathcal{F}}

% script G
\newcommand{\fang}{\mathcal{G}}

% script H
\newcommand{\fanh}{\mathcal{H}}

% script M
\newcommand{\fanm}{\mathcal{M}}

% script O
\newcommand{\fano}{\mathcal{O}}

% script P
\newcommand{\fanp}{\mathcal{P}}

% script R
\newcommand{\fanr}{\mathcal{R}}

% script S 
\newcommand{\fans}{\mathcal{S}}

% script T 
\newcommand{\fant}{\mathcal{T}}

% script U 
\newcommand{\fanu}{\mathcal{U}}

% script Z
\newcommand{\fanz}{\mathcal{Z}}

%*********************
% Special Constructs *
%*********************

% |x| over n
\newcommand{\overn}[1]{\frac{|#1|}{n}}

% ensembles and circuits
\newcommand{\ens}[2]{\{#1_#2\}_{#2 \in \nats}}

% probability
\newcommand{\pr}[3]{\text{Pr}_{#2 \in #3}[#1]}

% fiat-shamir signature scheme
\newcommand{\fs}[2]{SIG_{#2}(#1)}

% Merkle tree committment
\newcommand{\merk}[2]{TC_{f_#2}(#1)}

% ceiling
\newcommand{\ceil}[1]{\lceil #1 \rceil}

% floor
\newcommand{\floor}[1]{\lfloor #1 \rfloor}

% encoding into strings
\newcommand{\enc}[1]{\langle #1 \rangle}

% function definition
\newcommand{\func}[3]{#1 : #2 \to #3}



% Definitions
% Force proofs to end in a filled-in square rather than a hollow one.
\renewcommand{\qedsymbol}{$\blacksquare$}

% For dagger footnotes
\long\def\symbolfootnote[#1]#2{\begingroup%
\def\thefootnote{\fnsymbol{footnote}}\footnote[#1]{#2}\endgroup}

% When numbering sections, don't include the chapter number
\def\thesection{\arabic{section}} 

% A basic theorem environment
\theoremstyle{plain}
\newtheorem*{thm}{Theorem}

% A basic definition environment
\theoremstyle{definition}
\newtheorem*{defn}{Definition}

\theoremstyle{remark}
\newtheorem*{rem}{Remark}



\title{NOTES ON\\ On the (in)security of the Fiat-Shamir paradigm}
\author{Shafi Goldwasser and Yael Tauman Kalai}

\begin{document}

\maketitle

\section{Recap}
\subsection{Background}
The Fiat-Shamir transform, which we'll refer to as the 'FS transform' 
from now on, is a method for converting 3-round public-coin (or {\it
canonical}) id schemes into signature schemes using any efficiently
evaluable hash function ensemble $\fanh = \ens{\fanh}{n}$, $\fanh_n = 
\{\func{h_s}{\strs{*}}{\strs{n}} \}_{s \in \strs{n^2}}$. Although here we 
insist on $|s| = n^2$, any polynomial $p(n)$ would have done just as well. 
Also, when we say $\fanh$ is 'efficiently evaluable', all we mean is that 
there is a function $H$ s.t. $H(s,m) = h_s(m)$ for all $s,m \in \strs{*}$, 
and $H$ is computed by some deterministic TM $\fanm_H$ whose running time is 
polynomial in $|s|$ and $|m|$. The FS transform was first introduced, albeit 
implicitly, in \cite{fiat:fsparadigm}. 

Recall that a three-round public-coin id scheme $ID = (G,P,V)$ is simply a 
three-round id scheme where the verifier's move consists of his random bits, 
or 'coins'. In other words, the prover $P$ first sends the verifier $V$ a 
message $\alpha$, which we think of as a kind of 'commitment', then $V$ sends 
a random challenge $\beta$ to $P$, and finally $P$ responds with $\gamma$ 
(see Figure \ref{fig:threerounds}). 
\begin{figure}[h]
\begin{center}
\input{figures/threerounds.eepic}
\caption{A three-round public-coin id scheme $ID = (G,P,V)$}
\label{fig:threerounds}
\end{center}
\end{figure}

\noindent
Denote the signature scheme resulting 
from applying the FS transform to $ID$ with respect to $\fanh$ by 
$\fs{ID}{H} = (GEN_\fanh,SIGN_\fanh,VER_\fanh)$. 

On input $1^n$ (and random bits), 
$GEN_\fanh$ outputs $((PK,k),SK)$, where $(PK,SK) \gets G(1^n)$ and $k \in_R
\strs{n^2}$. In order to sign a message $m$, one must produce a transcript
$\sigma_m = (\alpha,\beta,\gamma)$ such that $\beta = h_k(\alpha,m)$ and
$V(PK,\sigma_m) = 1$. $SIGN_\fanh(PK,k,SK,m)$ can easily come up with such a
$\sigma_m$ by simulating $P(PK,SK,h_k(\alpha,m))$, where $\alpha$ is just the
first message $P(PK,SK)$ would normally send to $V(PK)$. 

Ideally, we'd like the FS transform to {\it preserve security}: given {\bf
any} secure canonical id scheme $ID$, there should exist {\bf some} ensemble
$\fanh$ such that $\fs{ID}{H}$ is a secure signature scheme.
It was shown in \cite{bellare:fiatshamirrom} that for any such $ID$, the 
``FS-type'' signature scheme one obtains by using a {\it truly random 
function} $\fano$ instead of an ensemble $\fanh$ is secure against
probabilistic polytime adversaries which have oracle access to $\fano$.
In other words, $\fs{ID}{H}$ is secure {\it in the random
oracle model}. This is taken to mean that the scheme has ``no structural
flaws'', since it's hard to conceive of a real-world attack which wouldn't
also work in the random oracle model.

However, security in the random model does not in fact imply real-world
security: Canetti, Goldreich, and Halevi have exhibited a (highly contrived) 
signature scheme in \cite{canetti:romfails} which is secure in the random 
oracle model, yet insecure no matter what ensemble $\fanh$ the random oracle 
$\fano$ is replaced with. Nevertheless, it was believed that signature schemes
obtained from secure canonical id schemes via the FS transform may still be
secure in the real world, since they are less ``perverse'' than the
unrealistic scheme constructed by Canetti et al.
 
Goldwasser and Tauman Kalai recently showed in 
\cite{goldwasser:fsparadigmfails} that the FS transform does {\bf not} in
fact preserve security. In other words, there exists {\bf some} secure
canonical id scheme $ID$ such that $\fs{ID}{H}$ is insecure for
{\bf every} $\fanh$. Although $ID$ is still quite ``perverse'', this result can
be viewed as another blow to the {\it random oracle methodology}, a popular 
approach to the disgn of cryptographic protocols first stated explicitly by 
Bellare and Rogaway in \cite{bellare:rompractical}. The idea behind
the random oracle methodology is to design a protocol which is secure in the 
random oracle model and ``hope'' that it remains secure even if the random 
oracle $\fano$ is replaced by an efficient hash function such as MD5 or SHA
(actually, we would need to use a family of functions rather than a single 
function, but that's a technical point). The utility of this approach stems 
from the apparent ease with which protocols secure in the random oracle model 
can be designed. 

Although both \cite{canetti:romfails} and
\cite{goldwasser:fsparadigmfails} effectively demonstrate the failure of the
random oracle methodology, the counterexamples used in those
papers rely on Micali's CS proofs \cite{micali:csproofs} and Barak's 
Universal Arguments \cite{barak:universalarguments}, respectively, and are
thus too unrealistic to rule out the methodology's practical usefulness. 
However, recent results such as \cite{nielsen:noromenc} and 
\cite{bellare:noromsig} suggest that the methodology probably isn't a very 
good idea even for practical constructions. In other words, ``security in the 
random oracle model'' isn't a particularly reliable indicator of real-world 
security.   

\subsection{A high-level overview of the proof}
The idea is to construct a secure canonical id scheme $ID$ such that
$\fs{ID}{H}$ is insecure for {\bf every} efficiently evaluable 
$\fanh$. In fact, we won't actually exhibit a single id scheme of this type; 
instead, we'll describe three different id schemes, $ID^1$, $ID^2$, and 
$ID^3$, such that one of them will provably have the desired properties.
The only complexity-theoretic assumption we'll need is that one-way functions 
exist. However, if they do not exist then secure signature schemes don't exist 
either, which means that the FS transform can't possibly yield secure 
signature schemes. In this sense, the failure of the FS paradigm is 
{\it unconditional}.

\subsection{What we've covered up to and including the June 17th
meeting}
\subsubsection{The FS paradigm fails if collision-resistant hash function 
ensembles {\emph don't} exist}
We first showed that the FS paradigm fails if collision-resistant hash 
function ensembles {\bf do not} exist. Recall that $\fanf = \ens{\fanf}{n}$ is 
said to be {\it collision-resistant} 
if the probability of finding $x,y \in \strs{*}$ s.t. $x \neq y$ and 
$f_s(x) = f_s(y)$, taken over $s \in \strs{n^2}$, is negligible in $n$.
This is hardly surprising, since it seems intuitively clear that $\fanh$ would 
need to be collision-resistant for $\fs{ID}{H}$ is to be secure. 

Remember that a function $\mu : \nats \to \nats$ is said to be {\it negligible}
if, for all $c \in \nats$, there exists an $n_0 \in \nats$ such that $\mu(n) < 
\frac{1}{n^c}$ for all $n \in \nats, n \geq n_0$. In other words, $\mu(n)$ 
goes to 0 asymptotically faster than the inverse of any polynomial in $n$. In
the sequel, we shall use $negl(n)$ to denote an anonymous negligible function.

We can now rewrite the above definition of collision resistance more formally, 
as follows: an ensemble $\fanf = \ens{\fanf}{n}$ is {\it collision-resistant} 
if, for all polysize circuit families $C = \ens{C}{i}$,
\[
p_C(n) = \pr{C_{n^2}(k) = \enc{x_1,x_2} : f_k(x_1) = f_k(x_2)}{k}{\strs{n^2}} =
negl(n).
\]  

\noindent
Here and elsewhere, $\enc{\cdot,\cdot}$ denotes an anonymous ``reasonable'' 
encoding of $\strs{*}\times \strs{*}$ into $\strs{*}$. More formally, a
``reasonable'' encoding scheme is simply a pair of deterministic polytime 
algorithms $(ENC,DEC)$ such that $DEC(ENC(x,y),0^{|x|}) = x$ and 
$DEC(ENC(x,y),1^{|y|}) = y$.

Let $SIG = (GEN,SIGN,VER)$ be some secure signature scheme; $SIG$ must exist 
under our assumption that one-way functions exist. Construct a canonical id 
scheme $ID = (G,P,V)$ as follows: the key generation algorithm
$G$ is identical to $GEN$. $P$'s first message is some fixed string, say the 
empty string $\lambda$, and $V(PK,\lambda,\beta,\gamma) = 1$ if and only if
$VER(PK,\beta,\gamma) = 1$, i.e. $\gamma$ is a legitimate signature of the
random challenge string $\beta$ with respect to the key pair $(PK,SK) \gets
G(1^n)$. Since impersonating $P$ would require an adversary to successfully 
sign a random challenge in $SIG$, the security of $ID$ follows from that of 
$SIG$. However, $\fs{ID}{H}$ is insecure for all $\fanh$, as we show below. 

Fix a hash function ensemble $\fanh = \ens{\fanh}{n}$. A polysize 
adversary $ADV = \{ADV_n\}_{n \in \nats}$ breaks the security of 
$\fs{ID}{H} = (GEN_\fanh, SIGN_\fanh, VER_\fanh)$ as follows: say that 
$((PK,k),SK)$ is generated by running $GEN_\fanh$ on $1^n$ together with some 
random bits, and $ADV_n$ is given $(PK,k)$. $ADV_n$ first finds two distinct 
strings $x$ and $y$ s.t. $h_k(x) = h_k(y)$, where $h_k \in \fanh_n$ is
uniquely specified by the key $k \in \strs{n^2}$. Since $\fanh$ is {\bf not} 
collision-resistant -- because no ensemble is, by assumption -- $ADV_n$ succeeds
in doing this with some non-negligible probability, which we denote by 
$p_{ADV}(n)$. $ADV_n$ then queries its signing oracle $SIGN_\fanh(SK)$ on $x$ 
to obtain a valid signature $\sigma_x = (\alpha, \beta,\gamma)$ of $x$, where
$\beta = h_k(x)$ because $h_k(\lambda,x) = h_k(x)$. Finally, $ADV_n$ outputs 
the pair $(y,\sigma_x)$, where $y$ is the message he's trying to sign and
$\sigma_x$ is its supposed signature. 

Since $h_k(x) = h_k(y)$, $\sigma_x$ is indeed a valid signature of
$y$ with respect to $((PK,k),SK)$. Therefore the probability that 
$VER(PK,k,y,\sigma_x) = 1$ is exactly the probability that $ADV_n$ found such 
a $y$ in the first place, i.e. $p_{ADV}(n)$. But $p_{ADV}(n)$ is 
non-negligible in $n$, by assumption, so $ADV$ breaks the security of 
$\fs{ID}{H}$, as promised.

\subsubsection{The FS paradigm fails if collision resistant hash function 
ensembles {\emph do} exist}
Assume that collision resistant hash function ensembles do exist, as is
widely believed, and let $\fanf = \ens{\fanf}{n}$, where
$\fanf_n = \{\func{f_s}{\strs{2n}}{\strs{n}}\}_{s \in \strs{n^2}}$, be one 
such ensemble. We will use $\fanf$ as part of a special {\it computationally 
binding} commitment scheme (we'll explain what we mean by this below), called 
a {\it Merkle tree commitment} \cite{merkle:tree}, which will enable us to
construct the id scheme we need. Let us first explain how tree committments 
work and what sort of properties they have.

Fix a security parameter $n$, and randomly choose a key $k \in \strs{n^2}$.
This key uniquely determines a function $f_k \in \fanf_n$,
$\func{f_k}{\strs{2n}}{\strs{n}}$. 

Consider any $x \in \strs{*}$ such that $|x| = n\cdot2^m$ for some $m \in 
\pints = \nats \setminus \!\{0\}$. We may view $x$ as being made of up of 
$2^m$ $n$-bit 'blocks': $x = x_1,\dots,x_{2^m}$, $|x_i| = n, 1 \leq i 
\leq 2^m$. A tree committment to $x$ under $f_k$, which we'll denote by
$\merk{x}{k}$, consists of the integer $m$ together with the label assigned to 
the root of the complete binary tree of height $m$ (recall that such a tree 
has exactly $2^m$ leaves) by the following recursive labelling:
\begin{itemize}
\item Base case: the leaves of the tree are labelled by the $n$-bit blocks of 
$x$. In other words, $label_x(leaf_i) = x_i$, $1 \leq i \leq 2^m$, where the
leaves are ordered from left to right.
\item Induction step: to label an internal node, concatenate the labels of its
two children together and apply $f_k$ to the resulting $2n$-bit string, 
i.e. $label_x(node) = f_k(label_x(left\_child(node))\circ 
label_x(right\_child(node)))$, where '$\circ$' denotes the string concatenation 
operator. 
\end{itemize}
Since $f_k$ maps $2n$-bit strings to $n$-bit strings, this procedure assigns
an $n$-bit label, $label_x(node)$, to every node in the tree as it works its 
way up towards the root. We call such a labelled complete binary tree {\it the
Merkle tree corresponding to $x$}. 

Recall that a complete binary tree with
$2^m$ leaves has $2^{m+1} - 1$ nodes in total. We can therefore number the 
nodes of the Merkle tree from 1 to $2^{m+1} - 1$, starting at the root and 
working down towards the leaves from left to right, so that the root has index
1 and the rightmost leaf has index $2^{m+1} - 1$.  

Finally, set 
\[
\merk{x}{k} = label_x(root) \circ [m]_2,
\] 
where '$\circ$' denotes the concatenation operator and
'$[m]_2$' denotes the binary representation of $m$. Note that no special 
separator is needed to distinguish between the two parts of $\merk{x}{k}$, 
since $|label_x(root)| = n$, with $n$ fixed and made pulbic
prior to the first application of the committment. 

Recall that $m = \log_2(\overn{x})$, so that 
\[
\ell = |[m]_2| = \ceil{\log_2(m)} = \ceil{\log_2(\log_2(\overn{x}))}
\] 
Although we've only looked at $x$'s whose length is linear in $n$ up till now, 
this means that even if $|x|$ were {\it doubly exponential} in $n$, e.g. 
$|x| = 2^{2^{c\cdot n}}$ for some $c \in \pints$, $k$ would still be linear in 
$n$. Thus 
\[
|\merk{x}{k}| = n + \ell = \fano(n)
\] 
for all $x$'s we are likely to ever care about. 

However, observe that although the {\it space} required to store strings which 
are exponentially long in the security parameter $n$ is linear in $n$, the 
{\it time} required to compute the Merkle tree committments to such strings is 
still exponential in $n$. This is because we must examine every bit of a
string in order to commit to it. 

So far, we have restricted our attention to $x$'s such that $|x| = n\cdot2^m$
for some $m \in \pints$. How do we commit to strings that are not of this 
(rather special) form? The important point to realize is that {\it any} $x 
\in \strs{*}$ can be padded to an $x'$ such that $|x'| = n\cdot2^m$ for some 
$m \in \pints$ in a uniquely decodable way: simply let $\ell = 
\ceil{\log_2(\overn{x})}$, $m = n\cdot2^{\ell} - |x| - 1$, and set 
$x' = 0^m1x$. Since $|x'| = (n\cdot2^{\ell} - |x| - 1) + 1 + |x| =
n\cdot2^{\ell}$, we have $\log_2(\overn{x'}) = \log_2(2^{\ell}) = \ell$,
a positive integer. Notice that $x$ should technically be padded with a
leading '1' before $m$ is computed, for otherwise we will have $m = -1$ for
$x$'s that are already of the right form. We can easily recover $x$ from $x'$ 
by stripping away all of the leading characters up to and including the first 
'1'. 

In light of this fact, we will henceforth assume (wlog) that 
$m = \log_2(\overn{x}) \in \pints$.

What about decommittment? The standard way to decommit to $x \in \strs{*}$ is
to simply send $x$ to the other party. Having received a supposed
committment $y = y_1 \circ y_2$, where $y_1 \in \strs{n}$ and 
$y_2 \in \strs{\ell}$, followed by $x \in \strs{*}$, that party will 
accept if and only if $|x| = n\cdot2^m$, where $m$ is the integer $y_2$ is the 
binary representation of, and $label_x(root) = y_1$. 

%It is now clear why we included $[m]_2$ as part of $\merk{x}{k}$: since 
%$|label_x(root)| = n$ for any length $x$, we need some mechanism for ensuring 
%that   
Now that we've specified both how to commit and how to decommit to strings 
using the Merkle tree committment scheme, it remains to explain what sort of 
properties it possesses. 

\medskip \noindent
{\bf CLAIM}: If $\fanf$ is collision resistant, then $TC_\fanf$, the Merkle 
tree committment with respect to $\fanf$, is {\it computationally binding}. 
That is, for every security parameter $n$, given a randomly chosen $k \in 
\strs{n^2}$, which uniquely determines a function $f_k \in \fanf_n$, it is 
infeasible to find two distinct strings of equal length whose Merkle tree 
committments with respect to $f_k$ coincide. 

More formally, for all polysize circuit families $C = \ens{C}{i}$, 
\[
q_C(n) = \pr{C_{n^2}(k) = \enc{x_1,x_2} : 
|x_1| = |x_2| \, \wedge \, x_1 \neq x_2 \, \wedge \, \merk{x_1}{k} = 
\merk{x_2}{k}}{k}{\strs{n^2}} = negl(n).
\]

\medskip \noindent
{\bf PROOF}: The proof uses the standard ``randomized reduction'' technique,
which Goldreich refers to as a 'reducibility argument' in
\cite{goldreich:foundations1}. 

Suppose that there exists a polysize circuit family $ADV = \ens{ADV}{i}$ which
breaks the computational unambiguity of $TC_{\fanf}$, so that  
$q_{ADV}(n) > \frac{1}{n^c}$ for some $c \in \pints$ and infinitely many 
$n$'s. We will use $ADV$ to construct a polysize circuit family $ADV'$ for 
which $p_{ADV'}(n) > \frac{1}{n^c}$, which means that $ADV'$ breaks the 
collision resistance of $\fanf$. Since $\fanf$ is collision-resistant 
by assumption, this is a contradiction.

Fix an $n$ such that $q_{ADV}(n) > \frac{1}{n^c}$. 

On input $k \in \strs{n^2}$, $ADV_{n^2}'$ first runs $ADV_{n^2}$ on
$k$ to obtain a pair of strings $\enc{x_1,x_2}$. It then constructs the Merkle
trees corresponding to $x_1$ and $x_2$, and compares the labels of the two
trees node by node, starting with node $2^{m} - 1$ (i.e. the rightmost node in 
the level directly above the leaves) and working up towards the root from
right to left. If $ADV'_{n^2}$ finds an index $i$ such that
$label_{x_1}(node_i)\neq label_{x_2}(node_i)$, it outputs $\enc{y_1,y_2}$, 
where $y_1 = label_{x_1}(left\_child(node_i))\circ 
label_{x_1}(right\_child(node_i))$ and 
$y_2 = label_{x_2}(left\_child(node_i))\circ 
label_{x_2}(right\_child(node_i))$, so that $|y_1| = |y_2| = 2n$.

What is the probability $p_{ADV'}(n)$ that $y_1 \neq y_2$, yet $f_k(y_1) = 
f_k(y_2)$? 

We know that
$x_1 \neq x_2$, $\merk{x_1}{k} = \merk{x_2}{k}$ with probability $q_{ADV}(n) >
\frac{1}{n^c}$. Let $x_1 = x_1^1\circ x_1^2 \circ \cdots \circ x_1^{2^{m}}$
and $x_2 = x_2^1 \circ x_2^2 \circ \cdots \circ x_2^{2^{m}}$, where $m \in 
\pints$ and $|x_1^i| = |x_2^i| = n, 1 \leq i \leq 2^{m}$.

%Let $ID = (G,P,V)$ be some secure canonical id scheme, say one obtained from a
%secure signature scheme, as outlined above. Our goal is to modify $V$ 
%to $V'$ in such a way that $ID' = (G,P,V')$ is still secure, yet 
%$\fs{ID'}{H}$ is insecure for all $\fanh$. The idea is to
%``compromise'' $V$'s code in a way that can't be exploited by an adversary 
%trying to impersonate $P$, yet gives away enough information to enable an 
%adversary to successfully forge signatures in $\fs{ID'}{H}$.

% References
\bibliographystyle{alpha}
\bibliography{$THESISDIR/share/random_oracles}
\end{document}
