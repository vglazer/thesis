\chapter{Uninstantiability}
\label{CH:Uninstantiability}

\section{The random oracle methodology and ``uninstantiable'' schemes}
\label{SEC:Methodology}
As originally proposed in \cite{bellare:rompractical} and applied in practice,
the Random Oracle Methodology involves taking a construction which is secure
in the ROM (usually under a standard hardness assumption) and
``instantiating'' the random oracle $\fanr$ using a ``cryptographically
strong'' hash function $h: \strs{*} \to \strs{n}$; whenever $\fanr$ is queried
on $m \in \strs{*}$, the answer is $h(m)$.  However, a hash function {\it
ensemble} $\fanh = \ens{\fanh}{n}$, $\fanh_n = \{h_s : \strs{*} \to
\strs{n}\}_{s \in \strs{n}}$ (see Chapter~\ref{CH:Preliminaries},
Section~\ref{SEC:Hash}) must strictly speaking be used instead. The reason is
that, as we'll see shortly, it isn't hard to come up with signature and
public-key encryption schemes (see sections \ref{SEC:Signatures} and
\ref{SEC:PKEPs} of Chapter~\ref{CH:Preliminaries}, respectively) which are
secure in the ROM yet become hopelessly insecure if $\fanr$ queries are
answered using a fixed function $h$. We therefore call a signature or
public-key encryption scheme {\it uninstantiable} if it is secure in the ROM
(possibly under some hardness assumption) yet insecure in the real world, no
matter what hash function ensemble $\fanh$ is used to instantiate the random
oracle $\fanr$.

Since random oracles are one-way (see Chapter~\ref{CH:Preliminaries},
Section~\ref{SEC:ROM}), the results of \cite{rompel:1waysigs} imply that
signature schemes which are secure in the ROM exist unconditionally\footnote{A
subtle but important technical point to note here is that both Rompel's
construction and his proof are of the ``black-box'' variety.}. Given a hash
function $h$, we can obtain a signature scheme which is secure in the ROM but
cannot be instantiated using $h$ by modifying any signature scheme which is
secure in the ROM as follows. Given a message $m \in \strs{*}$, the new signer
computes a signature $\sigma$ of $m$ as before and then checks whether
$\fanr(\bar{0}) = h(\bar{0})$. If so, he outputs $(pri,\sigma)$; otherwise, he
outputs $(\bar{0},\sigma)$. 
%(here we assume, without loss of generality, that $|pri| = n$). 
Given a message $m$ and a purported signature $\alpha$ of $m$, the new
verifier parses $\alpha$ as $(\beta,\gamma)$, where $|\beta| = n$, and then
checks whether $\gamma$ is a valid signature of $m$ as before.
%so that the verifier accepts any pair $(m,\sigma)$ signer , who is given a
%message $m \in \strs{*}$, reveals the private key $pri$ whenever
%$\fanr(\bar{0}) = h(\bar{0})$; notice that the above modification has no
%effect on the correctness of the scheme. 
Observe that our modification does not violate the correctness of the scheme,
since every signature output by the signer is accepted by the verifier,
whether $\fanr(\bar{0}) = h(\bar{0})$ or not. The modified scheme also
remains secure in the ROM, because the probability that $\fanr(\bar{0}) =
h(\bar{0})$ (taken over the randomness of $\fanr$) is $\frac{1}{2^n}$. 
%will learn $pri$ in this way is $\frac{1}{2^n}$.
%negligible in the security parameter $n$; specifically, it is
%at most $\frac{q}{2^n}$, where $q$ is the number of times $ADV$
%queries his signature (or decryption) oracle.  
However, once $\fanr$ is instantiated using $h$, all the forger has to do to
learn $pri$ --- thereby completely breaking the scheme's security --- is query
his signature oracle on some string (say $\lambda$ for concreteness).

The above approach can be readily adapted to yield a public-key encryption
scheme which is secure in the ROM, but cannot be instantiated using $h$. To
obtain such a scheme, simply take any public-key encryption scheme which is
secure in the ROM (as noted in Chapter~\ref{CH:Introduction}, in light of the
results of \cite{impagliazzo:nooneway} a hardness assumption of some sort will
almost certainly be necessary here) and modify it as follows. Given a message
$m \in \strs{n}$, the new encryptor computes an encryption $e$ of $m$ as
before and then  checks whether $\fanr(\bar{0}) = h(\bar{0})$. If so, he
outputs $(m,e)$; otherwise, he outputs $(\bar{0},e)$. Given a purported
encryption $\alpha$, the new decryptor first parses $\alpha$ as
$(\beta,\gamma)$, where $|\beta| = n$, then decrypts $\gamma$ as before.
Observe that our modification once again doesn't violate the correctness of
the scheme, since every encryption output by the encryptor is correctly
decrypted by the decryptor, whether $\fanr(\bar{0}) = h(\bar{0})$ or not. It
is easy to see that the modified scheme remains secure in the ROM, but becomes
completely insecure if $\fanr$ is instantiated using~$h$.
%all the forger has to do to
%obtain $pri$ (thereby completely breaking the scheme's security) is query his
%signature (or decryption) oracle.
% on some message.
%For example, let $SIG$ be a signature scheme which is secure in the ROM, and
%consider the following modification $SIG'$ of $SIG$. Given a message $m \in
%\strs{*}$, the new signer first checks whether $\fanr(m) = h(m)$. If so, he
%outputs the private key $pri$ as the signature of $m$; otherwise, he signs
%$m$ as before. $SIG'$ remains secure in the ROM, since the probability that a
%polynomial-time forger learns $pri$ is negligible in the security parameter
%$n$ (specifically, it is equal to $\frac{q}{2^n}$, where $q$ is the number of
%times the signature oracle $\fans$ is queried).  Given a particular
%efficiently-evaluable function $h: \strs{*} \to \strs{n}$, it is easy to come
%up with signature schemes which are secure in the ROM, yet become insecure
%when the random oracle $\fanr$ is ``instantiated'' using $h$ (i.e. $\fanr(s)$
%is set to $h(s)$ for all $s \in \strs{*}$). For instance, let $SIG =
%(GEN^\fanr,SIGN^\fanr,VER^\fanr)$ be a signature scheme which is secure in
%the ROM, and consider the following modification $SIG' =
%(GEN^\fanr,SIGN^\fanr,VER^\fanr)$ of $SIG$. The situation for public-key
%encryption schemes is similar, except that a hardness assumption of some sort
%is needed to show that the scheme is secure in the ROM.  Although this state
%of affairs may justly be viewed as a failure of the Random Oracle
%Methodology, we are willing to disregard such ``contrived'' or
%``pathological'' examples. Instead, we are interested in constructions which
%are secure in the ROM, yet insecure 

\section{The first uninstantiability result}
\label{SEC:Original}
Canetti, Goldreich and Halevi first showed that uninstantiable signature and
public-key encryption schemes exist in \cite{canetti:romfails}. 
%their seminal 1998 paper three landmark theorems about security in the random
%oracle model. The theorems underscore the limitations of the random oracle
%methodology, in effect ``separating'' ROM-security and ``real-world''
%security. 
Their key insight was that, for every hash function ensemble $\fanh =
\ens{\fanh}{n}$, $\fanh_n = \{h_s : \strs{*} \to \strs{n}\}_{s \in \strs{n}}$,
there exists a binary relation $R^\fanh = \bigcup_{n \in
\nats}\{(s,h_s(s))\}_{s \in \strs{n}}$ with the following two properties:
\begin{enumerate}[(i)]

\item There is a (deterministic) polynomial-time machine $M_\fanh$ which,
given any $s \in \strs{n}$, outputs an $x \in \strs{*}$ such that
$(x,h_s(x)) \in R^\fanh$.

\item For every probabilistic polynomial-time ``finder'' $F^\fanr$ who is
given $1^n$, the probability (taken over the random bits of $F^\fanr$ and the
randomness of $\fanr$) that $F^\fanr$ outputs an $x \in \strs{*}$ such that
$(x,\fanr(x)) \in R^\fanh$ is negligible in $n$.

\end{enumerate}
%Specifically, $R^\fanh = \bigcup_{n \in \nats}\{(s,h_s(s))\}_{s \in
%\strs{n}}$. 
$R^\fanh$ obviously satisfies property (i), since $M_\fanh$ can simply output
$s$ itself as $x$. To see that $R^\fanh$ satisfies property (ii), observe that
$(x,\fanr(x)) \in R^\fanh \Leftrightarrow \fanr(x) = h_x(x)$.  $F^\fanr$'s
success probability is therefore at most $\frac{q_\fanr}{2^n}$, where
$q_\fanr$ is the (polynomially bounded) number of times he queries $\fanr$.
%), which is negligible in $n$. 
Notice that $R^\fanh$ is also polynomial-time decidable in the following
sense: to determine whether $(x,y) \in R^\fanh$, one need only compute $y' =
h_x(x)$ (this can be done in polynomial time, because $\fanh$ is
efficiently evaluable) and check if $y = y'$.

Given any hash function ensemble $\fanh$, we can use $R^\fanh$ to obtain a
signature scheme which is secure in the ROM yet becomes insecure when the
random oracle $\fanr$ is instantiated using $\fanh$. The idea is to take a
signature scheme which is secure in the ROM (as pointed out in
Section~\ref{SEC:Methodology}, such schemes exist unconditionally) and modify
it as follows. Given a message $m \in \strs{*}$, the new signer computes a
signature $\sigma$ of $m$ as before and then checks whether $(m,\fanr(m)) \in
R^\fanh$ (this can be done in polynomial time, since $R^\fanh$ is
polynomial-time decidable). If so, he outputs $(pri,\sigma)$; otherwise, he
outputs $(\bar{0},\sigma)$. Given a message $m$ and a purported signature
$\alpha$ of $m$, the new verifier parses $\alpha$ as $(\beta,\gamma)$, where
$|\beta| = n$, and then checks whether $\gamma$ is a valid signature of $m$ as
before.
%If the message $m \in \strs{*}$ to be signed satisfies
%$(m,\fanr(m)) \in R^\fanh$ --- this can be determined in polynomial time,
%since $R^\fanh$ is polynomial-time decidable --- the new signer outputs the
%private key $pri$ as the signature of $m$; otherwise, he signs $m$ as before.
Observe that our modification does not violate the correctness of the scheme,
since every signature output by the signer is accepted by the verifier,
whether $(m,\fanr(m)) \in R^\fanh$ or not.  The modified scheme also remains
secure in the ROM, since property (ii) above guarantees that any forger has
only a negligible probability of finding an $m$ such that $(m,\fanr(m)) \in
R^\fanh$. However, once $\fanr$ is instantiated using $\fanh$, the forger, who
is given $s$, need only query his signature oracle $\fans$ on $s$ to obtain
$pri$. 
%Notice that, since $R^\fanh$ is polynomial-time decidable, the new signer
%still runs in polynomial time.

We next use diagonalization to go from schemes which cannot be instantiated
using some specific ensemble $\fanh$ to schemes which cannot be instantiated
using {\it any} ensemble. Recall from Section~\ref{SEC:Hash} of
Chapter~\ref{CH:Preliminaries} that every hash function ensemble $\fanh$ can
be identified with its polynomial-time ``evaluator'' Turing machine $M_\fanh$. 
%a (deterministic) polynomial-time Turing machine $M_\fanh$
%such that $M_\fanh(s,x) = h_s(x) \in \strs{n}$ for all $s \in \strs{n}$ and $x
%\in \strs{*}$. 
We can therefore effectively enumerate all hash function ensembles by
enumerating all polynomial-time Turing machines and padding or truncating
their output as necessary. Let $M_\fanu$ be the universal Turing machine doing
the enumerating, and denote the corresponding ``universal'' ensemble by $\fanu
= \ens{\fanu}{n}$. Since the running time of {\it every} polynomial-time
machine cannot be upper-bounded by a {\it single} polynomial, $M_\fanu$ will
need to run in ``slightly super-polynomial'' time, say 
$\fano(n^{\log n})$ for concreteness.
%Specifically, let $M_\fanu$ be a Turing machine which, given $\langle M
%\rangle,s \in \strs{n}$ and $x \in \strs{*}$, simulates $M$ on $(s,x)$ for
%some super-polynomial number of steps, say $(2n + |x|)^{\log_2(2n + |x|)}$, and
%pads or truncates the output so that it is of length exactly $2n$.
%and padding or
%truncating their output as necessary.
%, we use the ``universal'' or ``diagonal'' ensemble $\fanu = \ens{\fanu}{k}$,
%$\fanu_k = \{h_s^i : \strs{*} \to \strs{\frac{k}{2}}\}_{(i,s) \in \strs{k}}$.
%When run on $i,s \in \strs{n}$ and $x \in \strs{*}$, the corresponding
%``evaluator'' $M_\fanu$ simulates the $i^{th}$ Turing machine
%Denote the ensemble corresponding to $M_\fanu$ by $\fanu = \ens{\fanu}{k}$.
It is easy to see that when $\fanu$ is substituted for $\fanh$ in the above
construction, the resulting signature scheme is uninstantiable.  However, the
signer no longer runs in polynomial time, since to determine whether
$(m,\fanr(m)) \in R^\fanu$ he must effectively simulate $M_\fanu$. 
%(which runs in time $k^{\log_2k})$ 

Fortunately, the above difficulty can be overcome with the aid of Micali's
non-interactive CS proofs (\cite{micali:csproofs2}, \cite{micali:csproofs1}).
Let $M_\fanu'$ be a decider for $R^\fanu$.
% namely, given $(i,s,y)$, $M_\fanu'$
%accepts if and only if $y = h_s^i = M_\fanu(i,s)$. 
Instead of running $M_\fanu'$ directly to determine whether $(m,\fanr(m)) \in
R^\fanu$, the new signer parses $m$ as $(s,\pi)$ and checks if $\pi$ is a
valid CS proof that $M_\fanu'$ accepts $(s,\fanr(s))$ within $\fano(n^{\log
n})$ steps, where $n = |s| + |\fanr(s)|$. Since CS proofs can be verified very
efficiently, this only takes polynomial time. In the ROM, the scheme remains
secure because CS proofs are ``computationally sound''\footnote{Interestingly,
it is not known whether non-interactive CS proofs are computationally sound in
the ``real world'' under some reasonable complexity assumption.}, meaning that
it is infeasible to find a valid proof of a false statement. However, once
$\fanr$ is instantiated using some ensemble $\fanh$, the ``perfect
completeness'' property of CS proofs guarantees that a forger can compute a
valid $\pi$ in polynomial time.

%A non-interactive CS proof system is a pair $(P^\fanr,V^\fanr)$ of
%deterministic polynomial-time oracle Turing machines satisfying the following
%two properties: \begin{enumerate}[(i)] \item (Perfect Completeness) For every
%$(1^n, \langle M \rangle, x, t, \pi)$ such that $M$ accepts $x$ in $t$ steps
%Turing machine $M$ accepts $x \in \strs{*}$ in $t$ steps, then $V^\fanr$
%accepts  with probability 1, where the probability is taken  over the
%randomness of $\fanr$.  \item (Computational Soundness)
%\end{enumerate}
%Consider a Turing machine $M_\fanu'$ which accepts $(x,y)$ if and only if
%Instead of simulating $M_\fanu$ to determine whether $(m,\fanr(m)) \in
%R^\fanu$, the new signer parses $m$ as $(m',\pi)$ and reveals $pri$ only if
%$V^\fanr(1^n, \langle M_\fanu \rangle, m', $ accepts $\pi$.  For every
%$\langle M \rangle$ , $x \in \strs{*}$ and $1^t$ (where $\langle M \rangle$
%describes a Turing machine $M$ under some reasonable encoding) such that $M$
%accepts $x$ within $t$ steps, the prover $P^\fanr$ outputs a proof $\pi \in
%\strs{*}$ such that the following two conditions hold:
%, which are known to be ``computationally sound '' in the ROM.
%We can similarly construct a public-key encryption scheme which is secure in
%the ROM yet becomes insecure when $\fanr$ is instantiated using $\fanh$.
%However, in light of the results of \cite{impagliazzo:nooneway}, a hardness
%assumption of some sort will almost certainly be necessary in this case.
Just as in Section~\ref{SEC:Methodology}, the above approach can be readily
adapted to yield an uninstantiable public-key encryption scheme.

\section{A simple proof of the first result}
In \cite{maurer:indiffability}, Maurer, Renner and Holenstein introduced a new
type of reducibility, based on the concept of {\it indifferentiability}. 
%They then show that random oracles are not reducible (according to this new
%definition) to another type of primitive called an ``asynchronous beacon''. 
To motivate their definitions, they gave a simple
proof of the existence of uninstantiable signature and public-key encryption
schemes. We present a further simplified version of their argument below.
%a slightly modified version of their proof.

To obtain an uninstantiable signature scheme, modify any signature scheme
which is secure in the ROM (as pointed out in
Section~\ref{SEC:Methodology}, such schemes exist unconditionally) as follows.
Given a message $m \in \strs{*}$, the new signer first computes a signature
$\sigma$ of $m$ as before. He then parses $m$ as $(\langle M \rangle,1^t)$,
where $\langle M \rangle$ describes a (deterministic) Turing machine $M$ under
some reasonable encoding, and simulates $M$ on $\langle M \rangle$ for at most
$t$ steps. If $M$ outputs $\fanr(\langle M \rangle)$, the signer outputs
$(pri,\sigma)$; otherwise, he outputs $(\bar{0},\sigma)$.
%(as in Section~\ref{SEC:Methodology}, we assume without loss of generality that
%$|pri| = n$). 
Given a message $m$ and a purported signature $\alpha$ of $m$,
the new verifier parses $\alpha$ as $(\beta,\gamma)$, where $|\beta| =
n$, and then checks whether $\gamma$ is a valid signature of $m$ as before.
Observe that our modification does not violate the correctness of the scheme,
since every signature output by the signer is accepted by the verifier,
whether $M$ outputs $\fanr(\langle M \rangle)$ within $t$ steps or not. Also
note that the new signer runs in polynomial time, since simulating $M$ takes
time $\fano(t)$ and $t \leq |m|$.

To convince yourself that the modified scheme remains secure in the ROM,
consider a function family $\fanf = \ens{f}{t}$ where each $f_t: \strs{*} \to
\strs{*}$ is defined by 
\begin{equation*}
f_t(\langle M \rangle) =
\begin{cases}
M(\langle M \rangle) & \text{ if $M$ halts within $t$ steps } \\
\lambda & \text{ otherwise}
\end{cases}
\end{equation*}
To learn $pri$ using the ``trapdoor'' we have built into the scheme, a forger
would effectively need to find a $t \in \nats$ and an $x \in \strs{*}$ such
that $f_t(x) = \fanr(x)$. His probability of finding such a pair $(t,x)$ is at
most $\frac{q_\fanr}{2^n}$ (where $q_\fanr$ is the number of times he queries
$\fanr$), which is negligible in $n$.
%since no forger can predict $\fanr(\bar{0})$ with probability greater than
%$\frac{1}{2^n}$.

Once $\fanr$ is instantiated using a hash function ensemble $\fanh$, however,
it becomes trivial to completely break the scheme's security. Recall that,
because $\fanh$ is efficiently evaluable, there exists a (deterministic)
polynomial-time Turing machine $M_\fanh$ such that $M_\fanh(s,x) = h_s(x)$ for
all $s \in \strs{n}$ and $x \in \strs{*}$ (see Chapter~\ref{CH:Preliminaries},
Section~\ref{SEC:Hash}). Let $M_{h_s}$ denote $M_\fanh$ with some particular
$s$ ``hard-coded'' into it, so that $M_{h_s}(x) = M_\fanh(s,x)$ for all $x \in
\strs{*}$, and suppose that $n^c$ is an upper bound on the running time of
$M_{h_s}$. When given input $\langle M_{h_s} \rangle$, $M_{h_s}$ halts within
$n^c$ steps and outputs $h_s(\langle M_{h_s} \rangle)$. The forger, who is
given $s$, need therefore only query his signature oracle on $(\langle M_{h_s}
\rangle, 1^{n^c})$ to learn $pri$. $\blacksquare$

Just as in Section~\ref{SEC:Methodology}, the above approach can be readily
adapted to yield an uninstantiable public-key encryption scheme.

\section{An uninstantiability result for Fiat-Shamir signature schemes}
\label{SEC:FiatShamir}
The artificiality of \cite{canetti:romfails}'s constructions left open the
possibility that ``reasonable'' signature schemes which are secure in the ROM,
and in particular Fiat-Shamir signature schemes, can in fact be instantiated
using appropriate hash function ensembles. 
%(perhaps dependent on the particular scheme in question).  
However, in \cite{goldwasser:fsparadigmfails} Goldwasser and Tauman-Kalai
showed that there exist uninstantiable Fiat-Shamir signature schemes. It must
be remarked that Goldwasser and Tauman-Kalai's construction is, if anything,
even more contrived than those of \cite{canetti:romfails}. Barak and
Goldreich's Universal Arguments (\cite{barak:universalarguments}) are used in
place of Micali's CS proofs, and Merkle trees (\cite{merkle:tree}) also make
an appearance. Most distressingly, the proof itself has a highly
non-constructive, tree-like structure: rather than demonstrate that a single
(albeit unnatural) Fiat-Shamir scheme is uninstantiable, Goldwasser and
Tauman-Kalai exhibit three such schemes, one of which must be uninstantiable.
Nonetheless, from a purely theoretical standpoint, Goldwasser and
Tauman-Kalai's result deals a severe blow to the validity of the so-called
Fiat-Shamir paradigm (see Chapter~\ref{CH:Fiatshamir},
Section~\ref{SEC:Overview}).

%Let $SIG = (GEN^\fanr,SIGN^\fanr,VER^\fanr)$ be a signature scheme which is
%secure in the ROM; no hardness assumptions are necessary here, since secure
%signature schemes exist if one-way functions do (\cite{rompel:1waysigs}), and
%$\fanr$ is one-way in a strong, information-theoretic sense.  CGH use
%Micali's CS proofs (see \cite{micali:csproofs1} and \cite{micali:csproofs2})
%to construct a signature scheme and a public-key encryption scheme which are
%secure (in the chosen-message and chosen-ciphertext or CCA2 sense,
%respectively) in the random oracle model yet insecure in the ``real world'',
%no matter what efficiently computable \emph{hash function ensemble} is used
%to implement the oracle. Such schemes are now commonly called
%``uninstantiable''. This is a somewhat startling result, since CS proofs are
%only known to possess their crucial ``proof-of-knowledge'' property in the
%random oracle model.  The reason one must resort to using cumbersome
%\emph{function ensembles} instead of good old functions is as follows. Given
%any suitable candidate function $f$, it is easy to come up with pathological
%schemes which are secure in the random oracle model, yet insecure when oracle
%queries are replaced with evaluations of $f$. Note that these schemes are
%concocted \emph{after} $f$ is fixed.  The downside of CGH's schemes is that
%they are prohibitively inefficient. It would be nice to have examples of
%\emph{efficient, practical} uninstantiable signature and encryption schemes.
%Additionally, CGH demonstrate that ``correlation-intractable'' ensembles do
%not exist (in the ``real world''). Informally, this means that there is no
%\emph{single} efficiently evaluable hash function ensemble which has
%\emph{all} of the nice information-hiding properties enjoyed by a truly
%random function (i.e. a random oracle). More formally, we call a binary
%relation $\fanr$ ``evasive'' if the probability that a PPT\footnote{{\bf
%P}robabilistic {\bf P}oly {\bf T}ime} OTM\footnote{{\bf O}racle {\bf T}uring
%{\bf M}achine} adversary $A^{\fano}$ outputs a pair $(x,y) \in \fanr$ given
%$1^n$ is negligible in the security parameter $n$, where the probability is
%taken over the coins of $A$ and the randomness of the oracle $\fano$. An
%ensemble $\fanf = \ens{\fanf}{n}$, where $\fanf_n = \{f_s: \strs{*} \to
%\strs{\ell(n)}\}_{s \in \strs{n}}$ for some appropriate length function
%$\ell: \nats \to \nats$ ($\ell(|x|) \equiv |x|$, for example) , is said to be
%\emph{correlation intractable} if, for every evasive relation $\fanr$, the
%probability that a PPT TM adversary $ADV$ which is given a randomly chosen
%``key'' $s \in \strs{n}$ outputs a string $x$ such that $(x,f_s(x)) \in
%\fanr$ is negligible in $n$. Notice that a correlation-intractable ensemble
%$\fanf$ must ``work well'' for \emph{every} evasive relation. Therefore
%\emph{specialized} ensembles which only work for some particular evasive
%relation $\fanr$ might still exist.  Again informally, the non-existence of
%correlation-intractable ensembles implies that, if there is to be any hope of
%retaining security in the ``real world'', the function ensemble used to
%``instantiate'' the random oracle must somehow depend on the functionality we
%have in mind (because no one ensemble works in all cases). 

%Although CGH exhibited an
%``uninstantiable'' signature scheme secure in the ROM, their construction was
%extremely contrived, making their impossibility result less convincing. In
%practice, people were far more interested in the uninstantiability of
%practical Fiat-Shamir (\cite{fiat:fsparadigm}) and Bellare \& Rogaway
%(\cite{bellare:signrsa}) type schemes. Goldwasser and Tauman-Kalai exhibit an
%uninstantiable Fiat-Shamir signature scheme in
%\cite{goldwasser:fsparadigmfails}.
%Unfortunately, their construction, which heavily relies on universal arguments
%(see \cite{barak:universalarguments}), is again impractical. In fact, GTK's
%proof isn't even fully constructive, since their case-by-case analysis
%only establishes that one of three specified signature schemes must be
%uninstantiable, without indicating which one it might be. This time, universal
%arguments play a role similar to the one played by CS proofs in
%\cite{canetti:romfails}.
%\section{Some recent results}
%\item \cite{nielsen:separating} (2002): In the context of secure asynchrous
%multiparty computation (MPC) as defined in \cite{canetti:univ}, Nielsen shows
%that secure non-interactive non-committing encryption is possible in the
%random oracle model but not in the ``real-world'', nor in a weaker model he
%calls the ``non-programmable'' ROM. His results suggest that, in the MPC
%context at least, it is the ``programmability'' of the random oracle that
%cannot be duplicated in the ``real world''.
%\begin{itemize}
%\item \cite{canetti:lengthromfails} (2004): CGH extend the negative result of
%\cite{canetti:romfails} by exhibiting a \emph{stateless} uninstantiable
%signature scheme which is secure in the ROM. Moreover, the attack which
%demonstrates the scheme's uninstantiability requires only signatures of
%``short messages'', i.e. ones of length polylogarithmic in the security
%parameter $n$ ($\log^4n$, to be exact). The attack employed in
%\cite{canetti:romfails} required signatures of messages longer than the
%public key.  Instead of CS proofs, the new construction uses a novel
%interactive proof system based on \emph{Merkle trees} \cite{Merkle:tree},
%which incidentally also played a crucial role in
%\cite{goldwasser:fsparadigmfails}. 
%\item \cite{maurer:indiffability} (2004): Maurer, Renner and Holenstein
%introduce a new type of indistinguishability for cryptosystems called
%\emph{indifferentiability}, complete with a corresponding notion of
%reducibility. They then show that a random oracle is not reducible to a
%primitive called an \emph{asynchronous beacon}. This yields a very clean and
%simple proof of the theorem from \cite{canetti:romfails} which asserts that
%there exist uninstantiable public-key cryptosystems secure in the ROM.  \item
%\cite{bellare:noromsig} (2004): Asymmetric (i.e. public-key) encryption
%schemes are typically used in a ``hybrid'' fashion. Suppose we are in the
%so-called Public-Key Infrastructure (PKI) setting, where  If $A$ wants to
%communicate with $B$ securely, he randomly chooses a \emph{session key} $k$,
%encrypts $k$ under $B$'s public key $pub$ and sends the resulting ciphertext
%$c$ to $B$. $B$ then uses his private key $pri$ to decrypt $c$, obtaining
%$k$.  $A$ and $B$ are now in possession of a shared private key $k$, and can
%therefore communicate using a symmetric (i.e. shared-private-key) encryption
%scheme.
%Why doesn't $A$ just encrypt his messages to $B$ directly, using the
%assymetric encryption scheme? One reason is \emph{efficiency}. Symmetric
%encryption schemes typically require only fast operations such as binary
%exclusive-or, whereas asymmetric schemes rely on expensive operations like
%modular exponentiation. Another reason is \emph{security}. Symmetric schemes
%are provably secure if the underlying function generators (usually DES or
%AES) are pseudorandom, whereas the security of asymmetric schemes relies on a
%plethora of complexity-theoretic conjectures, ranging from the plausible
%(factoring and discrete log) to the not-so-plausible (elliptic curves).
%Bellare, Boldyreva and Palacio define what it means for an asymmetric scheme
%to be ``IND-CCA-preserving'', and then show that Hash ElGamal, a particular
%modification of the ElGamal asymmetric encryption scheme
%(\cite{elgamal:dlsigs}), is IND-CCA-preserving in the random oracle model,
%yet yields an insecure system when the oracle is instantiated using any
%efficiently-evaluable hash function ensemble. Notice that an asymmetric
%scheme can be IND-CCA-preserving without actually being IND-CCA (i.e. CCA2)
%secure, since it is only used to transmit randomly chosen messages. In
%particular, the scheme may be \emph{deterministic}.  \end{itemize}
