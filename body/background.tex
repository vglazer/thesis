\chapter{Definitions and Basic Results}

\section{Signatures}
\subsection{Definitions}
\subsection{Basic Results}
\begin{itemize}

\item \cite{goldwasser:signatures1} (1984): Goldwasser, Micali and Rivest
exhibit the first signature scheme which is provably secure against adaptive
chosen message attack ('secure' from now on) under a plausible complexity-theoretic assumption. Here the
assumption is that ``claw-free'' trapdoor permutations exist, which is true if
factoring is hard. Unfortunately, GMR's ``signature tree'' construction,
whereby messages are associated with the leaves of a binary tree, and each node on
the path from a leaf to the root must be authenticated in order to sign a
particular message, is not practical. The full version (\cite{goldwasser:signatures2}) appeared in SICOMP
in 1988.

\item \cite{bellare:trapdoorsigs1} (1988): Bellare and Micali show how to
obtain a secure signature scheme from any \emph{one-way} trapdoor permutation
(no need for claw-freeness this time). Their construction is also
of the inefficient, ``signature tree'' type. The full version (\cite{bellare:trapdoorsigs2}) 
appeared in JACM in 1992. 

\item \cite{naor:universal} (1989): Naor and Yung show how to construct
secure signature schemes from weakly collision-resistant hash
functions, and describe how the latter can be obtained from 1-1 one-way
functions. In other words, they show that if 1-1 one-way functions exist then
secure signature schemes exist. This is a two-fold improvement on Bellare and
Micali's result: first, there's no need for the function to have a trap door
(there are notoriously few natural examples of trapdoor one-way functions).
Second, there's no need for the function to be \emph{onto}, so that the domain
and codomain need not be identical (although collisions are still not
allowed). Alas, Naor and Yung's construction is once again impractical. It is
also not stateless, which imposes additional storage requirements.

\item \cite{rompel:1waysigs} (1990): Building on the work of Naor and Yung, Rompel
shows how to obtain weakly collision-resistant hash functions from ordinary (not necessarily injective)
one-way functions, thereby demonstrating that secure signature
schemes exist if one-way functions exist. This necessary condition is also
sufficient, since any secure signature gives rise to a one-way function.
Romplel's result is mainly of theoretical importance, since it effectively
reuses Naor and Yung's impractical construction.
\end{itemize}


\section{ID schemes}
\subsection{Definitions}
\begin{itemize}
\item See pages 270-274 of \cite{goldreich:foundations1} for some basic definitions.
\end{itemize}

\subsection{Basic Results}

\section{Encryption}

\subsection{Definitions}
\subsection{Basic Results}
\begin{itemize}
\item \cite{goldwasser:probenc1} (1982): Goldwasser and Micali define
semantic security and propose a practical cryptosystem which is semantically
secure if quadratic residuosity is hard. The full version
(\cite{goldwasser:probenc2}) appeared in JCSS in 1984.

\item \cite{rackoff:cca2} (1991): Rackoff and Simon define CCA2 security and
exhibit the first CCA2-secure cryptosystem. However, their construction relies on
non-interactive zero-knowledge proofs of knowledge (see
\cite{blum:noninterzerok}, \cite{bellare:noninterzk}, \cite{naor:cca1},
\cite{feige:multnizk}, and pages 298-311 of \cite{goldreich:foundations1}) and is hence impractical.
Another drawback of the RS construction is that it requires the presence
of a trusted central authority that issues keys to all parties (i.e. a
Public Key Infrastructure).

\item \cite{dolev:nonmalleable1} (1991) Dolev, Dwork, and Naor independently construct
another, more complicted CCA2 secure cryptosystem (in the context of
non-malleability). Their cryptosystem uses non-interactive zero-knowledge
proofs of knowledge as well and also isn't practical, but it does not rely on
the presence of a public key infrastructure. The full version of this
paper (\cite{dolev:nonmalleable2}) did not appear until 2000. 

\item \cite{sahai:nizkcca} (1999): Sahai introduces the concept of
\emph{non-malleable} NIZKs, and uses them to convert
Naor and Yung's CCA1-secure cryptosystem (\cite{naor:cca1}) into a CCA2-secure
one. If one-way functions exist, then every NIZK can be (efficiently) converted
to a non-malleable one. At a high level, Sahai's modification of NY's scheme is much simpler
and more elegant than Dolev, Dwork and Naor's CCA2 cryptosystem
(\cite{dolev:nonmalleable1}). However, the details are still fairly messy. The
cryptosystem is secure provided one-way trapdoor permutations exist.

From a zero-knowledge perspective, one shortcoming of Sahai's construction is
that it is only \emph{one-time simulation sound}, i.e. it is malleable if
the adversary has access to more than one NIZK proof.

\item \cite{desantis:robustnizk} (2001): De Santis, Di Crescenzo, Ostrovsky,
Persiano and Sahai introduce \emph{many-time simulation sound} NIZKs, 
both strengthening and simplifying Sahai's construction (\cite{sahai:nizkcca}). DDOPS's new NIZKs also
yield a simpler, more elegant NY-type CCA2-secure cryptosystem.

\item \cite{lindell:simplercca2} (2003): Yehuda Lindell applies the ideas
of \cite{desantis:robustnizk} to one-time simulation sound NIZKs (which are
sufficient for CCA2-secure cryptosystems) in order to come up with an
even simpler NY-type CCA2-secure cryptosystem. His is the nicest cryptosystem
whose CCA2 security rests on general assumptions -- the existence of one-way
traproor permutations, to be exact -- to date.
\end{itemize}

