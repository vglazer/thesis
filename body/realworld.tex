%\chapter{Practical signature and encryption schemes secure in the ``real word''}
\chapter{A taste of ``real-word'' security}
\label{CH:Realworld}

In the following two sections, we briefly survey a number of practical
signature schemes and public-key encryption schemes which are  secure in the
``real world'' (as opposed to in the ROM) under either standard or
nonstandard-yet-quite-plausible hardness assumptions.

\section{Signature schemes}
\label{SEC:SigSchemes}

\begin{itemize}

\item In \cite{dwork:fastsigs}, Dwork and Naor proposed a practical signature
scheme which is secure --- that is, secure against existential forgery under
adaptive chosen-message attack (see Chapter~\ref{CH:Preliminaries},
Section~\ref{SEC:Signatures}) --- under the standard ``RSA assumption''.
%provided that ``claw-free trapdoor permutations'' exist (they do if factoring
%is hard). 
Informally, the RSA assumption says that the following problem is
hard: given a modulus $n = pq$ where $p$ and $q$
are random primes, a random $y \in \fanz_n^*$ and a random exponent $e$  
%chosen randomly from among all integers
%between $0$ and $n-1$ which are 
relatively prime to $(p-1)(q-1)$, find an $x \in \fanz_n^*$ such that $x^e
\equiv y \mod n$. Although it can be shown that this problem would be easy if
$p$ and $q$ were given explicitly, 
%as opposed to just their product $n$, 
it is not known whether factoring $n$ can be reduced to finding $x$. While their
construction is conceptually similar to the ``authentification trees'' of
\cite{goldwasser:signatures2}, Dwork and Naor's use of ``bushy trees'' of high
degree and small depth rather than binary trees significantly improves
efficiency: for some reasonable settings of the security parameters, signing
requires only four tree authentications.  A significant drawback of their
construction is that all signers and verifiers must share two lists, one
consisting of random integers and the other of random primes.

\item In \cite{cramer:fastsigs}, Cramer and Damg\aa rd described an improved
version of Dwork and Naor's signature scheme (\cite{dwork:fastsigs}), secure
under the same assumptions. In this new version, signers and verifiers need
only share a single list consisting of random primes. 

\item In \cite{gennaro:fastsigs}, Gennaro, Halevi and Rabin presented a rather
efficient ``hash-and-invert'' signature scheme secure under the
nonstandard-yet-quite-plausible ``strong RSA assumption''. Informally, the
strong RSA assumption says that the following problem is hard: given a modulus
$n = pq$ where $p$ and $q$ are random primes and a random $y \in \fanz_n^*$,
find an $x \in \fanz_n^*$ and an exponent $1 < e < n$ relatively prime to
$(p-1)(q-1)$ such that $x^e \equiv y \mod n$; notice that, unlike in the RSA
problem, here $e$ is allowed to depend on $y$. Gennaro, Halevi and Rabin's
scheme makes use of ``collision-resistant chameleon hash functions'', which
exist if factoring is hard. For typical settings of the security parameters,
it is more than twice as efficient as Cramer and Damg\aa rd's scheme
(\cite{cramer:fastsigs}). 
%One interesting aspect of GHR's construction is that
%security is proven in two stages. First, the construction is shown to be
%secure in the random oracle model. Next, the random oracle is ``instantiated''
%using a hash function ensemble, and a reasonable (if non-standard)
%security-preserving assumption on the ensemble is identified. This is
%precisely the approach to eliminating random oracles suggested by Canetti in
%\cite{canetti:hideinfo} and refined by Canetti, Micciancio and Reingold in
%\cite{canetti:perfecthash}.
%(in 1997 and 1998, respectively).

\item In \cite{cramer:signatures2}, Cramer and Shoup presented another
efficient ``hash-and-invert'' signature scheme secure under the ``strong RSA
assumption''. Their scheme builds on that of Cramer and Damg\aa rd
(\cite{cramer:fastsigs}) and is considerably simpler and potentially more
efficient than Gennaro, Halevi and Rabin's (\cite{gennaro:fastsigs}). Instead
of ``collision-resistant chameleon hash functions'', it makes use of
``universal one-way hash functions'' (\cite{naor:universal}), which exist if
one-way functions do.
%(recall from Chapter~\ref{CH:Preliminaries}, Section~\ref{SEC:Signatures} that 
%if one-way functions do not exist, then %neither do secure signature schemes).  
Interestingly, a slight modification of the scheme can be shown to be secure
in the ROM under the ordinary RSA assumption.

%Note
%that if one-way functions do not exist then neither do secure signature
%schemes, since any secure signature scheme gives rise to a corresponding
%one-way function.

\item In \cite{fischlin:cramershoup}, Fischlin described an improved version
of Cramer and Shoup's signature scheme (\cite{cramer:signatures2}), again
secure under the ``strong RSA assumption''. Signing is about thirty percent
faster in this new version, and verification is somewhat faster as well.
Also, the length of the signatures is nearly halved. 
%whereas public keys are slightly longer. 

%\item In \cite{boneh:shortsigsnorom}, Boneh and Boyen exhibited an efficient
%signature scheme secure under the nonstandard-yet-quite-plausible ``strong
%Computational Diffie-Hellman assumption'', which may be viewed as a 
%discrete logarithm analogue of the ``strong RSA assumption''. 
%%used to prove the security of
%%\cite{gennaro:fastsigs} and \cite{cramer:signatures2}.  
%Their scheme is related to the one presented in \cite{boneh:shortsigs}, and
%similarly yields signatures which are almost half the size of DSS signatures
%with the same level of security.
%%though not quite as short as the ones in \cite{boneh:shortsigs}.  
%Boneh and Boyen also describe a very efficient modification of the current
%scheme with greatly reduced signature length which is secure in the random
%oracle model (under the same strong assumption).

\end{itemize}

\section{Public-key encryption schemes}
\label{SEC:EncSchemes}

\begin{itemize}
\item In \cite{cramer:cca2secure}, Ronald Cramer and Victor Shoup proposed the
first practical public-key encryption scheme which is secure --- that is,
secure against adaptive chosen-ciphertext attack or CCA2-secure (see
Chapter~\ref{CH:Preliminaries}, Section~\ref{SEC:PKEPs}) --- under a fairly
standard hardness assumption, namely the ``Decisional Diffie-Hellman
assumption''. Informally, the Decisional Diffie-Hellman assumption (often
abbreviated as the DDH assumption) holds for a cyclic multiplicative group $G$
of prime order $q$ (say a subgroup of $\fanz_p^*$, where $p > q$ is some
prime) if, given $g^u \in G$ and $g^v \in G$ for randomly chosen $u,v \in
\{1,\ldots,q\}$ (where $g \in G$ is some fixed generator of $G$), it is hard
to distinguish $g^{uv} \in G$ from $g^r \in G$ for a randomly chosen $r \in
\{1,\ldots,q\}$. While the Computational Diffie-Hellman or CDH assumption (see
Chapter~\ref{CH:Encryption}, Section~\ref{SEC:DHIES}) asserts that it is hard
to compute {\it all of} $g^{uv}$, the DDH assumption effectively asserts that
it is hard to compute {\it any bit of} $g^{uv}$.
%Recall that the {\it Computational}
%Diffie-Hellman assumption says that $g^{xy}$ is hard to compute from $g^x$ and
%$g^y$, where $g$ is some fixed generator of $\fanz_p^*$ and $x, y \in
%\{0,\ldots,p-1\}$ are chosen randomly (see Chapter~\ref{CH:Encryption},
%Section~\ref{SEC:EncOverview}).

\item In \cite{shoup:hedge}, Victor Shoup presented a ``hybrid'' encryption
scheme which makes use of a ``pseudorandom number generator'', a
collision-resistant hash function (see Chapter~\ref{CH:Preliminaries},
Section~\ref{SEC:Hash})
and a ``key encapsulation scheme''; the latter is based on the Cramer-Shoup
encryption scheme (\cite{cramer:cca2secure}). A key encapsulation scheme is
essentially just a public-key encryption scheme whose security is only
guaranteed when the messages being encrypted are random (private keys, for
example).  The new scheme is somewhat more efficient than
\cite{cramer:cca2secure} and is secure under the fairly standard DDH
assumption.  Interestingly, it is also secure in the ROM under the standard
CDH assumption. 
%This new cryptosystem is no less efficient than
%the original Cramer-Shoup one; in fact, it is slightly more efficient.


%\item In \cite{cramer:hashproofs}, Cramer and Shoup introduce the notion
%of a universal hash proof system for NP-languages (this is essentially a special kind
%of non-interactive zero-knowledge proof system), and show how they can be used
%to construct CCA2-secure public key cryptosystems. They construct universal
%hash proof systems for languages related both to Pallier's Decisional Quadratic
%Residuosity assumption (\cite{paillier:decisionqr}) and the classical Quadratic Residuosity assumption.
%These yield two new public key cryptosystems secure in the standard model, the
%first (based on DQR) fairly practical while the second (based on QR) less so.
%They also show that the original Cramer-Shoup scheme can be viewed as being
%based on a DDH-related hash proof system.

%\item In \cite{camenisch:verifiable1}, Camenisch and Shoup modify the
%decisional QR cryptosystem from \cite{cramer:hashproofs} in order to
%accomodate an efficient protocol for verifiable encryption of discrete
%logarithms (while still preserving CCA2 security). This is the ``full''
%version; see \cite{camenisch:verifiable2} for the conference version.

%\item In \cite{elkind:unified}, Elkind and Sahai present a framework,
%called the ``oblivious decryptors model'', which unifies two-key, Naor-Yung-type
%(\cite{naor:cca1}) and ``universal hash proofs''-based, Cramer-Shoup-type
%(\cite{cramer:cca2secure}) practical CCA2-secure cryptosystems using Sahai's 
%simulation-sound NIZKs (\cite{sahai:nizkcca}).

%\item In \cite{gennaro:framework1}, Gennaro and Lindell show how to use
%``smooth projective hash functions'' (\cite{cramer:hashproofs}) in order to implement secure
%password-based authenticated key exchange. The hardness assumptions here are
%either the DDH or the quadratic (and, more generally n)-residuosity assumption.
%The setting is as follows: a bunch of people want to communicate securely, and
%each \emph{pair} shares a ``low-entropy'' password, i.e. a string randomly
%chosen from some small (say of size polynomial in the security parameter) set.
%They want to somehow agree on a high-entropy session key that would enable
%them to communicate using highly-efficient symmetric encryption protocols, but
%the network is under the complete control of the adversary. All parties also
%share a ``common reference string''. This is the ``full'' version. See
%\cite{gennaro:framework2} for the conference version.

%\item In \cite{canetti:semanticibe}, Canetti, Halevi and Katz exhibit a
%practical identity-based public-key cryptosystem which is \emph{semantically} secure in the
%standard model under the bilinear Decisional Diffie-Hellman assumption. In
%fact, it also have some forward security properties, but we don't care about
%these.

\item In \cite{kurosawa:newparadigm}, Kurosawa and Desmedt described a new
hybrid encryption scheme based on \cite{shoup:hedge}. The scheme is somewhat
more efficient (it saves one exponentiation and produces shorter encryptions)
and is again secure under the DDH assumption.  Kurosawa and Desmedt's key
insight was to notice that the underlying key encapsulation scheme need not be
CCA2-secure in order for the overall hybrid scheme to be CCA2-secure. However,
their proof requires the additional assumption that both the ``key derivation
function'' and the MAC used by the hybrid scheme are secure in a strong,
information-theoretic sense. In particular, the key to be exchanged must be
statistically close to random, precluding the use of pseudorandom number
generators.
%They then exhibited a more efficient encapsulation
%scheme which, although not itself CCA2-secure, possesses sufficient
%security for the overall scheme to be CCA2-secure.

\item In \cite{gennaro:noteonkd}, Shoup and Gennaro 
%pointed out that Kurosawa
%and Desmedt's (\cite{kurosawa:newparadigm}) proof of security introduced an
%additional assumption that both the ``key derivation function'' and the MAC
%used 
%in the symmetric (private-key) stage of 
%by the hybrid scheme are secure in a strong, information-theoretic sense. 
%In particular, the shared private key $k$ is
%required to be {\it statistically close} to random, and opposed to merely
%computationally indistinguishable from random. 
%Such strongly secure primitives
%are both less efficient and uncommon in practice. 
used the technique of ``deferred analysis'' to demonstrate that Kurosawa and
Desmedt's hybrid scheme (\cite{kurosawa:newparadigm}) is in fact secure under
the DDH assumption provided that both the ``key derivation function'' and the
MAC are secure in the ordinary, computational sense.
%security which uses ``deferred analysis'' to lift the additional security
%requirements.
%Additionally, SG's findings effectively improve the efficiency of the scheme,
%since the key can be derived using PRGs.

%\item In \cite{boneh:ibenorom}, Boneh and Boyen exhibit an identity-based
%public-key encryption scheme which is CCA2-secure in the standard model under the
%bilinear Diffie-Hellman assumption. Their construction is based on
%\cite{boneh:selectid}. Unfortunately, BB's construction is \emph{not} practical. 

%\item In \cite{canetti:cca2fromibe}, Canetti, Halevi and Katz show how to
%efficiently convert semantically secure identity-based encryption schemes into
%CCA2 secure ones. The resulting cryptosystem is somewhat less efficient than
%the underlying IBE one, however, because of the use of one-time signatures.

%\item In \cite{boneh:betterchk}, Boneh and Katz show that MACs (in
%conjunction with ``key encapsulation'') can be used
%to eliminate one-time signatures from CHK's construction
%(\cite{canetti:cca2fromibe}), so that the efficiency of the resulting CCA2-secure scheme is 
%roughly equivalent to that of the underlying IBE scheme. Very efficient
%schemes (of roughly the same efficiency as the Cramer-Shoup-type scheme
%described in \cite{kurosawa:newparadigm}) of this type are secure if the
%bilinear decisional Diffie-Hellman problem is hard.

\end{itemize}

