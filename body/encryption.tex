%\chapter{Public-key Encryption Schemes Secure in the ROM}
\chapter{A public-key encryption scheme CCA2-secure in the ROM}
\label{CH:Encryption}

%\section{Overview of Results}
%\subsection{Factoring-based}
%\subsubsection{RSA}
%\subsubsection{Quadratic and N-residues}
%\subsection{Discrete Log-based}
%\subsubsection{Computational Diffie-Hellman}
%\subsubsection{Decisional Diffie-Hellman}
%\subsubsection{Bilinear Diffie-Hellman}
\section{Public-key encryption in the ROM: an overview}
\label{SEC:EncOverview}
%\begin{itemize}

%\subsection{In the beginning}
%\item 
In \cite{bellare:rompractical}, Bellare and Rogaway proposed the following
public-key encryption scheme, which they showed to be CCA2-secure in the ROM (see Sections
\ref{SEC:PKEPs} and \ref{SEC:ROM} of Chapter~\ref{CH:Preliminaries} for the
relevant definitions).  Let 
%$\fang: \strs{*} \to \strs{n}$ and $\fanh:
%\strs{*} \to \strs{k}$ be random oracles, where $n$ and $k \leq n$ are 
%security parameters of the scheme, and let 
$(G,f,f')$ be a trapdoor permutation 
%for more on trapdoor permutations, 
(see Section~\ref{SEC:Trapdoor} of Chapter~\ref{CH:Preliminaries}). The key
generator $GEN^\fanr$ simulates $G$ to obtain a pair of keys $(k,k')$ and sets
$pub = k$, $pri = k'$. Let $n$ be the security parameter. To encrypt a message
$m \in \strs{n}$, $ENC^\fanr_k$ chooses $r \in \strs{n}$ randomly, computes $y
= f_k(r)$ and sets $e_m$ to $(y,\fanr(r)\oplus m, \fanr(r,m))$.  Given a
purported encryption $e = (\alpha,\beta,\gamma)$, $DEC_{k'}^\fanr$ computes $r
= f'_{k'}(\alpha)$ and sets $m = \beta \oplus \fanr(r)$. If $\gamma \neq
\fanr(r,m)$, he outputs $\perp$, indicating a failure to decrypt; otherwise,
he outputs $m$ as the decryption of $e$. Intuitively, $r$ is hard to find
because $f_k$ is hard to invert. Together with the randomness of $\fanr$, this
implies that $\fanr(r) \oplus m$ yields no information about $m$, which
guarantees semantic security. The ``authentification code'' $\fanr(r,m)$
ensures that encryptions are difficult to ``malleate''. 

%\subsection{The OAEP saga}
One drawback of the above scheme is that encryptions are of length about $3n$,
or roughly three times the length of the message being encrypted. To address
this issue, Bellare and Rogaway introduced the highly influential Optimal
Asymmetric Encryption Padding or OAEP scheme
in \cite{bellare:oaep}, a simplified version\footnote{The real scheme makes
use of two independent random oracles, $\fang$ and $\fanh$, and has three
security parameters.} of which is described next.  Let $\fanf = (G,f,f')$ be a
trapdoor permutation. As before, the key generator $GEN^\fanr$ simulates $G$
to obtain a pair of keys $(k,k')$ and sets $pub = k$, $pri = k'$. Let $n$ be
the security parameter. To simplify the presentation, it will be convenient to
assume that $f_k$ and $f'_{k'}$ map $2n$ bits to $2n$ bits (as opposed to $n$
bits to $n$ bits). To encrypt a message $m \in \strs{n}$, $ENC^\fanr_k$
chooses $r \in \strs{n}$ randomly, computes $\rho(m) = (m \oplus \fanr(r), r
\oplus \fanr(m \oplus \fanr(r)))$ and sets $e_m = f_k(\rho(m))$; $\rho$ is
sometimes called the {\it padding function}. Given a purported encryption
$\alpha \in \strs{2n}$, $DEC^\fanr_{k'}$ computes $(\beta,\gamma) =
f_{k'}'(\alpha)$, where $|\beta| = |\gamma| = n$, and sets $r = \gamma \oplus
\fanr(\beta)$. He then outputs $m = \beta \oplus \fanr(r) \in \strs{n}$ as the
decryption of $\alpha$; notice that every $\alpha$ is a valid encryption of
{\it some} $m$, so that $DEC_{k'}^\fanr$ never outputs $\perp$. This scheme
was shown to be ``plaintext aware'' in \cite{bellare:oaep}, where it was also
claimed (without proof) that plaintext awareness implies ``security against
chosen-ciphertext attack'' (it's not entirely clear whether the authors had
CCA1 or CCA2 security in mind).

After Bleichenbacher showed in \cite{bleichenbacher:pkcsbroken} that version
1.5 of RSA Security's PKCS~\#1 standard (\cite{rsa:pkcs15}) is vulnerable to
chosen-ciphertext attack, RSA-OAEP (a concrete implementation of the OAEP
scheme where the role of $\fanf$ is played by the RSA function) served as the
basis for version 2.0 of the standard (\cite{rsa:pkcs20}). RSA-OAEP was
subsequently also incorporated into IEEE's public-key cryptography standard,
IEEE P1363-2000 (\cite{ieee:p1363}).  However, in \cite{shoup:oaepbad} Shoup
pointed out that, although OAEP is indeed {\it CCA1}-secure, there is an
(additional, not random) oracle relative to which $\fanf$ remains one-way but
OAEP fails to be {\it CCA2}-secure. Since standard
``black-box'' security reductions relativize --- that is, hold relative to {\it
every} oracle --- any reduction from inverting $\fanf$ to breaking the CCA2
security of OAEP would therefore have to be ``non-black-box'', meaning that it
would need to somehow depend on the specifics of $\fanf$. Shoup also proposed a
modification of OAEP, called OAEP+, which he proved to be CCA2-secure in the
ROM. In \cite{fujisaki:rsaoaepok1}, Fujisaki, Okamoto, Pointcheval and Stern
proved that OAEP is CCA2-secure under the stronger assumption that $\fanf$ is
``partial domain one-way'', as opposed to ``full domain one-way'' or simply
one-way (also see \cite{fujisaki:rsaoaepok2} for the journal version). Since
RSA is ``random self-reducible'' --- loosely, this means that being able to
invert it on a large fraction of the inputs allows one to invert it on every
single input --- it is ``partial domain one-way''  if and only if it is
one-way.  Thus, Shoup's result notwithstanding, RSA-OAEP is in fact
CCA2-secure under the RSA assumption.

In \cite{boneh:simpleoaep}, Boneh observed that the OAEP padding function
$\rho$ may be viewed as two rounds of a ``Feistel network'' and proposed two
simpler, more elegant single-round padding functions. When used in conjunction
with either RSA or Rabin's modular squaring function, these new paddings yield
encryption schemes which are CCA2-secure in the ROM (under the assumption
that RSA is hard to invert and factoring is hard, respectively).
Interestingly, Boneh recommends using the Rabin function in preference to
RSA where his paddings are concerned, since it has better ``reduction
efficiency''.

%There hasn't been a great deal of work on discrete logarithm-based
%encryption schemes which are CCA2-secure in the ROM for the following reason. 
In \cite{cramer:cca2secure}, Cramer and Shoup described an efficient,
practical public-key encryption scheme (its efficiency is comparable to that
of RSA-OAEP) which is CCA2-secure in the ``real world'' (and hence also in the
ROM) under the non-standard-yet-highly-plausible ``Decisional Diffie-Hellman''
assumption.  This important result is perhaps the main reason why there hasn't been
a great deal of work done on discrete logarithm-based public-key encryption schemes which are
CCA2-secure in the ROM.
%comes up with a greatly
%simplified OAEP-type padding which is also CCA2 secure in the random
%oracle model. He also advocates using the Rabin (squaring) function rather
%than RSA, since the reduction from factoring to breaking the chosen ciphertext
%security is \emph{tighter} in that case. In other words, an adversary that breaks the CCA2 security
%of the cryptosystem can be converted into a factoring algorithm with very
%little overhead.
%
%\cite{bellare:oaep}, implies CCA1 security, it does not imply CCA2 security.
%Oracle Methodology and proposed a public-key encryption scheme which is CCA2-secure 
%in the ROM. This scheme later served as the basis for OAEP.                            
%
%\item 
%In \cite{bellare:oaep}, Bellare and Rogaway proposed OAEP, a cryptosystem
%which is purportedly CCA2 secure in the random oracle model if the RSA
%function is one-way.  
%
%\item 
%In \cite{bellare:minroms}, Bellare and Rogaway propose two new
%cryptosystems purportedly CCA2-secure in the random oracle model, one (OAEP-like) based on the hardness 
%of RSA and the other on the hardness of the computational Diffie-Hellman problem. 
% 
%\item 
%\subsection{A sampling of other schemes}
%In \cite{tsiounis:elgamal}, Tsiounis and Yung show that the ElGamal
%encryption scheme (\cite{elgamal:dlsigs}) is semantically secure in the ``real
%world'' under the \emph{decisional} Diffie-Hellman assumption. They also exhibit an 
%extension of the ElGamal scheme which is CCA2-secure in the random oracle model under
%the decisional Diffie-Hellman assumption and a special assumption about
%the unforgeability of Schnorr signatures (\cite{schnorr:discretelogfs}).
% 
%\item 
%In \cite{fujisaki:enhance}, Fujisaki and Okamoto show how to
%efficiently convert any semantically secure public-key cryptosystem into one that is CCA2-secure
%in the random oracle model.
% 
%\item \cite{fujisaki:integration} (1999): Fujisaki and Okamoto show how to
%combine a public-key cryptosystem which is ``one-way secure'' with a
%an indistinguishability-secure private-key cryptosystem to form a public-key
%cryptosystem which is CCA2-secure in the random oracle model. The public-key
%cryptosystem must additionally satisfy a technical condition called
%``$\gamma$-uniformity''. Informally, there must be ``enough'' distinct ciphertexts for every plaintext. 
%``One-way security'' is a non-standard, weak notion of security. A 
%cryptosystem is said to be ``one-way secure'' if no efficient adversary which is given a properly generated 
%public key and a random ciphertext 
%(i.e. the encryption of a random plaintext) can correctly decrypt it
%with significant probability.
%
%\item 
%In \cite{shoup:oaepbad}, Shoup points out that Bellare and Rogaway
%did \emph{not} in fact show that OAEP is CCA2 secure in the random oracle model.
%Rather, they showed it is ``plaintext aware''. However, the particular
%definition of ``plaintext-awareness'' used in their paper is \emph{not}
%equivalent to chosen ciphertext security. Shoup proposes an alternative
%padding scheme, dubbed OAEP+, which \emph{is} CCA2 secure in the random oracle
%model. Note that Shoup did not \emph{prove} that OAEP is insecure.
%Nevertheless, he did show that OAEP is insecure relative to an oracle. This
%rules out the existence of a standard ``black-box'' reduction from the one-wayness of the
%underlying trapdoor permutation to the chosen ciphertext security of the
%resulting cryptosystem.
% 
%\item 
%In \cite{fujisaki:rsaoaepok}, Expanding on Shoup's work, Fujisaki,
%Okamoto, Pointcheval, and Stern show that RSA-OAEP \emph{is} in fact CCA2 secure in the random oracle model.
%However, unlike Bellare and Rogaway's, their proof makes
%crucial use of the algebraic properties of the RSA function.
%\item 
%
%\item 
%In \cite{boneh:ibeweil1}, Dan Boneh and Matt Franklin describe an
%\emph{identity-based} encryption scheme which is CCA2 secure in the random oracle
%model, provided an elliptic curves analogue of the computational Diffie-Hellman
%problem is hard. The full version 
%(\cite{boneh:ibeweil2}) appeared in SICOMP
%in 2003.
% 
%\item 
%In \cite{cocks:iberomquad}, Cocks presents an identity-based
%encryption scheme which is CCA2-secure in the random oracle model if
%quadratic residuosity is hard.
%
%\end{itemize}

\section{The original scheme}
\label{SEC:DHIES}
In \cite{bellare:minroms}, Bellare and Rogaway proposed a discrete
logarithm-based public-key encryption scheme called 
%allegedly CCA2-secure in the ROM: an OAEP-like RSA-based scheme (see
%Section~\ref{SEC:EncOverview} for more on OAEP and its security) and a
%discrete logarithm-based scheme called 
the Diffie-Hellman Integrated Encryption Scheme or DHIES, which they claimed
is CCA2-secure in the ROM under the ``Computational Diffie-Hellman''
assumption (a standard discrete log-type hardness assumption). However, in
\cite{abdalla:dhies1} Abdalla, Bellare and Rogaway conceded that DHIES is
unlikely to be CCA2-secure in the ROM under the above assumption (see also
\cite{abdalla:dhies2}). Instead, they proved that DHIES is CCA2-secure in
the ``real world'' under a strong, non-standard Diffie-Hellman-type
assumption called the ``Hash Diffie-Hellman'' assumption;  their new focus on
``real-world'' security (as opposed to security in the ROM) was likely a
response to Cramer and Shoup's 1998 discovery (\cite{cramer:cca2secure}) of a
practical public-key encryption scheme which is CCA2-secure in the ``real
world'' under the non-standard-yet-plausible ``Decisional Diffie-Hellman''
assumption. Although its security rests on a rather less believable
assumption, DHIES is somewhat more efficient than Cramer and Shoup's scheme.
We now give an informal description of the DHIES encryption scheme.
%In Section~\ref{SEC:Ourscheme}, was
%show how to add redundancy to DHIES in order to ensure that it is indeed
%CCA2-secure in the ROM under the ``Computational Diffie-Hellman'' assumption.

%The scheme 
%is loosely based on the one suggested by ElGamal in
%\cite{elgamal:dlsigs} and 
%utilizes a secure MAC $(MAC,VER)$ and a secure shared private-key
%cryptosystem $(ENC,DEC)$ as follows. 

Let $G$ be a cyclic multiplicative group of order $p-1$, where $p$ is some
$n$-bit prime ($n$ being the security parameter). For concreteness, 
%it might help to 
think of $G$ as $\fanz^{*}_p = \{1,2,\ldots,p - 1\}$ (here the group
operation is multiplication mod $p$). We will need to assume that membership in
$G$ is efficiently testable, which is the case for $\fanz^*_p$. It will also
sometimes be convenient to treat the elements of $G$ as strings over
$\strs{}$, say via their binary encoding.  

Fix a generator $g \in G$, so that $G = \{g, g^2, \ldots, g^{p-1}\}$; 
$g$ is implicitly given to all participants, as are $p$ and $1^n$.
Informally, the Computational Diffie-Hellman assumption (often abbreviated as
the CDH assumption) with respect to $G$ says that $g^{uv} \in G$ is hard to
compute from $g^u \in G$ and $g^v \in G$.  Formally, the CDH assumption holds
for $G$ if, for every probabilistic polynomial-time adversary $A$ who is given
$g^u \in G$ and $g^v \in G$ for randomly chosen $u,v \in \{1, \ldots, p-1\}$,
the probability $p_A(n)$ that $A$ outputs $g^{uv} \in G$ is negligible; here
$p_A(n)$ is taken over the random bits of $A$ as well as the choice of $u$ and
$v$.  

We will need a secure MAC $\fanm = (SIGN,VER)$ (see
Section~\ref{SEC:Signatures} of Chapter~\ref{CH:Preliminaries}) and a secure
``private-key encryption scheme'' $\fane = (ENC,DEC)$ (the latter haven't
actually been formally defined in this thesis). $\fanm$ must also be
``non-malleable'' in the following sense: given a signature $\sigma$ of some
message $m$, it is infeasible to find another signature $\sigma' \neq \sigma$
of $m$.
%Although we haven't formally defined security for private-key encryption
%schemes in this thesis, 
Both secure non-malleable MACs and secure ``private-key encryption schemes''
can be implemented using ``pseudorandom function generators''
(\cite{goldreich:pseudorandom2}), which exist if one-way functions (see
Section~\ref{SEC:Oneway} of Chapter~\ref{CH:Preliminaries}) do
(\cite{goldreich:pseudorandom1}, \cite{goldreich:pseudorandom3}).   

To generate a matching public/private key pair, randomly choose a $v \in
\{1,\ldots,p-1\}$ and set $pub = g^v \in G$, $pri = v$. 

To encrypt a message $m \in \strs{n}$ given a public key $g^v$, 
%proceed as follows. 
first randomly choose $u \in \{1,\ldots,p-1\}$ and compute $g^u
\in G$, $(g^v)^u = g^{uv} \in G$. Next, query the random oracle $\fanr$ on
$(g^u, g^{uv})$ (notice that here we are treating elements of $G$ as binary
strings) to obtain two keys $k_1,k_2 \in \strs{n}$ (here we assume for
convenience that $\fanr : \strs{*} \to \strs{2n}$). Finally, compute $s =
ENC_{k_1}(m)$, $t = SIGN_{k_2}(s)$ and output $e_m = (g^u,s,t)$ as the
encryption of $m$. 

To decrypt a purported ciphertext $e = (\alpha,\beta,\gamma)$ given a private
key $v$, proceed as follows. If $\alpha \notin G$, output $\perp$, indicating
a failure to decrypt. Otherwise, compute $\alpha^v \in G$ and query $\fanr$ on
$(\alpha,\alpha^v)$ to obtain $k_1, k_2 \in \strs{n}$. If
$VER_{k_2}(\beta,\gamma) = 0$, output $\perp$. Otherwise, output
$DEC_{k_1}(\beta)$ as the decryption of $e$.

%Bellare and Rogaway claimed in \cite{bellare:minroms} that the above
%cryptosystem is secure against adaptive chosen ciphertext attack (i.e.
%CCA2-secure) in the random oracle model if the computational Diffie-Hellman
%problem is hard for $G$, and both $(MAC,VER)$ and $(ENC,DEC)$ are secure in
%the appropriate sense.  Here is some --- ultimately incorrect, as it turns
%out --- intuition as to why DHIES should be CCA2-secure in the ROM under the
%Computational Diffie-Hellman assumption, which basically says that it's hard
%to compute (all of) $g^{uv} \in G$ given $g^u \in G$ and $g^v \in G$ for
%randomly chosen $u,v \in \{1, \ldots, p-1\}$. Recall that the adversary $ADV$
%chooses two $n$-bit messages, $m_0$ and $m_1$, and then sees the encryption
%$e_b = (g^u,s,t)$ of $m_b$ for a randomly chosen $b \in \strs{}$.  Since
%$\fane$ and $\fanm$ are both secure, the only way If the adversary doesn't
%know the keys $k_1$ and $k_2$ then the chosen-message security of the MAC
%guarantees non-malleability and the indistinguishability security of the
%shared private key cryptosystem ensures semantic security. Since $\fanr$ is
%truly random, the only way to obtain $k_1$ and $k_2$ with significant
%probability is to compute $\alpha^v$ given $\alpha$ and $g^v$, i.e. to solve
%the Diffie-Hellman problem for $G$. 

%Although this informal argument is intuitively appealing, Abdalla, Bellare
%and Rogaway conceded in \cite{abdalla:dhies1} that the cryptosystem most
%likely cannot be formally proven secure, even in the random oracle model (see
%\cite{abdalla:dhies2} for the full version and \cite{abdalla:dhaes} for an
%earlier draft).  
%
%We propose a modified version of the cryptosystem and prove that it is CCA2
%secure in the random oracle model.

Next, we briefly argue why DHIES is unlikely to be CCA2-secure in the ROM
(without actually {\it proving} that it isn't).  The standard way to
demonstrate that a public-key encryption scheme is CCA2-secure is to show that
it is both semantically secure (see Chapter~\ref{CH:Preliminaries},
Section~\ref{SEC:PKEPs}) and ``plaintext-aware''. 
%; since we are working in the ROM, here both must be shown in the ROM. 
Informally, a public-key encryption scheme is plaintext-aware if the
decryption oracle $\fand$ is useless to the CCA2 adversary $ADV$, meaning that
$ADV$ is only able to get $\fand$ to decrypt ciphertexts he could have
decrypted himself. 
%($\fand$ outputs $\perp$ for the rest of the ciphertexts).
More formally, we say that a public-key encryption scheme is {\it
plaintext-aware in the ROM} if, for every probabilistic polynomial-time
adversary $ADV^{\fanr,\fand}$, there is a corresponding probabilistic
polynomial-time adversary $A^\fanr$ such that the difference between the
success probabilities of $ADV^{\fanr,\fand}$ and $A^\fanr$ 
%(taken over their random bits, the randomness of $\fanr$ and the choice of
%$pub$) 
is negligible (in $n$). $A^{\fanr}$ essentially simulates
$ADV^{\fanr,\fand}$, computing $\fand$'s answers himself 
%(with some negligible error) 
based on $ADV^{\fanr,\fand}$'s view. Below, we provide evidence that DHIES
fails to have this property.

Recall that $ADV^{\fanr,\fand}$ is given $g^v \in G$, chooses a pair of
messages $m_0,m_1 \in \strs{n}$ and then sees an encryption $e_b = (g^u, s =
ENC_{k_1}(m_b), t = SIGN_{k_2}(s))$, where $k_1k_2 = \fanr(g^u, g^{uv})$ and 
$b \in \strs{}$, $u \in \{1,\ldots,p-1\}$ are chosen randomly.
Since $ADV^{\fanr,\fand}$ is not allowed to query $\fand$ on $e_b$, he must
modify at least one of $g^u$, $s$ and $t$. 

If $ADV^{\fanr,\fand}$ queries $\fand$ on $e' = (g^{u'}, s', t')$, where
either $s' \neq s$ or $t' \neq t$ (or both), $A^\fanr$ can safely respond with
$\perp$. This is almost certainly the right answer, because in order for $e'$
to be a valid ciphertext $ADV^{\fanr,\fand}$ would need to either sign a new
message (if $s' \neq s$), or come up with another signature of an old message
(if $s' = s$); the former is infeasible because $\fanm$ is secure, whereas the
latter is infeasible because $\fanm$ is non-malleable.
%Similarly, if he queries $\fand$ on $(g^u, s', t)$ for some $s'
%\neq s$ then the answer is almost certainly $\perp$, because $(ENC,DEC)$ is a
%secure private-key encryption scheme by assumption. Querying $\fand$ on
%$(\alpha, s, t)$ for any $\alpha \notin G$ would also result in an answer of
%$\perp$. 
If, on the other hand, $ADV^{\fanr,\fand}$ queries $\fand$ on $(g^{u'},s,t)$
for some $u' \neq u$, the correct answer is again almost certainly $\perp$,
provided that
%Although $A^{\fanr}$ would like to give
%$\perp$ as the answer, $\fand$ will actually reveal $m_b$ if 
$g^{u'v} \neq g^{uv}$. Although we haven't done so, we could ensure that
is the case by stipulating that $|G| = q$ for some prime $q$.
%Since $A^\fanr$ has no information about
%$b$, the difference between his and $ADV^{\fanr,\fand}$'s success
%probabilities certainly wouldn't be negligible in this case. This means that
%our proof of plaintext-awareness will break down if it is possible for
%$ADV^{\fanr,\fand}$ to find 
%a $u' \neq u$ such that $g^{u'v} = g^{uv}$.
%In other words, $ADV^{\fanr,\fand}$ need only find 
%an $\alpha = g^{u'} \in G$ such that $\alpha^v = g^{uv}$ (note that he need
%not even know the value of $u'$ explicitly), which isn't ruled out by the
%Computational Diffie-Hellman assumption.  

However, as we shall now see, $\fand$ allows $ADV^{\fanr,\fand}$ to determine,
given any $\alpha,\beta \in G$, whether $\alpha^v = \beta$ (recall that
$ADV^{\fanr,\fand}$ does {\it not} know $v$). First, $ADV^{\fanr,\fand}$
computes $k_1k_2 = \fanr(\alpha,\beta)$ and creates an encryption $e' =
(\alpha, s' = ENC_{k_1}(m), t' = SIGN_{k_2}(s'))$ of some message $m \in
\strs{n}$, say $\bar{0}$ for definiteness. Next, he queries $\fand$ on $e'$.
It is easy to show that $\fand(e') = m$ (as opposed to $\perp$) if and only if
$\alpha^v = \beta$. Since it is by no means clear how $A^\fanr$ would emulate
such a functionality, a stronger assumption than CDH is apparently required to
ensure that DHIES is CCA2-secure in the ROM. 
%However, it should be stressed
%that the above reasoning doesn't conclusively establish that DHIES isn't
%CCA2-secure in the ROM under the CDH assumption, because some other proof of
%security which doesn't rely on demonstrating plaintext-awareness could exist
%in principle.

\section{Our modification}
\label{SEC:Ourscheme}
We now describe a modified version of DHIES, called DHIES+, which is provably
secure in the ROM under the CDH assumption. Although in what follows we only
show how to encrypt a single bit in order to simplify the presentation, our
scheme can be easily extended to encrypt $n$-bit messages with the aid of a
secure MAC and a secure private-key encryption scheme.

Let $G$ be a cyclic multiplicative group of order $q$, where $q$ is some
$n$-bit prime ($n$ being the security parameter). For concreteness, think of
$G$ as a subgroup of $\fanz_p^*$, where $p > q$ is some other prime. Notice
that here, unlike in Section~\ref{SEC:DHIES}, $|G|$ is prime. This will be
important in Section~\ref{SEC:Plaintext} below. Fix a generator $g \in G$, so
that $G = \{g, g^2, \ldots, g^q\}$; $g$ is once again implicitly given to all
participants, as are $q$ and $1^n$.

To generate a matching public/private key pair, randomly choose a $v \in
\{1,\ldots,q\}$ and set $pub = g^v \in G$, $pri = v$. 

To encrypt a bit $b$ given a public key $g^v$, first randomly choose $u \in
\{1,\ldots,q\}$ and compute $g^u \in G$, $(g^v)^u = g^{uv} \in G$. Next,
query $\fanr$ to obtain $s_0s_1r = \fanr(g^{uv})$, where $s_0,s_1,r \in
\strs{n}$ (here we assume for convenience that $\fanr: \strs{*} \to
\strs{3n}$).  Finally, compute $t = u \oplus r$ and output $e_b = (g^u, s_b,
t)$ as the encryption of $b$. 

To decrypt a purported ciphertext $e = (\alpha,\beta,\gamma)$ given a private
key $v$, proceed as follows:
\begin{enumerate}
\item If $\alpha \notin G$, output $\perp$, indicating
a failure to decrypt. 
\item Compute $\alpha^v \in G$ and query $\fanr$ to
obtain $s_0s_1r = \fanr(\alpha^v)$.
\item Set $u = \gamma \oplus r$ and compute $g^u \in G$.  If
$\alpha \neq g^u$, output $\perp$. 
\item If $\beta \notin \{s_0,s_1\}$, output
$\perp$. Otherwise, output a $b$ such that $\beta = s_b$ as the decryption of
$e$. 
\end{enumerate}
\begin{rem}
Notice that there is a small probability ($\frac{1}{2^n}$,
to be exact) that $s_0 = s_1$, in which case we won't be able to decrypt $e$
correctly. This unlikely occurrence can be avoided by making the scheme
slightly more complicated, but we won't go into the details here.
\end{rem}
We will prove that DHIES+ is CCA2-secure in the ROM in two stages. First,
we'll show that it is semantically secure in the ROM under the CDH
assumption. Next, we'll show that it is plaintext-aware in the ROM.

\subsection{Semantic security}
\label{SEC:Semantic}
Intuitively, DHIES+ is semantically secure in the ROM under the CDH assumption
because $s_b$ provides the adversary with no information about $b$ unless he
can compute $g^{uv} \in G$, which is infeasible if the Diffie-Hellman problem
is hard for $G$. More formally, suppose that $ADV^\fanr$ is a probabilistic
polynomial-time adversary who breaks the semantic security (see
Chapter~\ref{CH:Preliminaries}, Section~\ref{SEC:PKEPs}) of DHIES+ in the ROM. 
%and denote his
%(non-negligible) success probability by $p_{ADV}(n)$; the latter  
Since in this case there are only two possible plaintexts (namely $0$ and
$1$), there is no need to let $ADV^\fanr$ choose $m_0$ and $m_1$. Instead, he
simply gets a public key $g^v \in G$ and an encryption $e_b = (g^{u}, s_b, t)$
of a randomly chosen bit $b$, where $s_0s_1r = \fanr(g^{uv})$ and $t = u
\oplus r$.  Denote $ADV^\fanr$'s success probability by
$p_{ADV}(n)$ and the total number of times he queries $\fanr$ during his
execution by $c(n)$; $p_{ADV}(n)$ is taken over $ADV^\fanr$'s random bits,
as well as the randomness of $\fanr$ and the choice of $b$, $u$ and $v$. Since
$ADV^\fanr$ runs in strict polynomial time, $c(n) \leq n^c$ for some $c$. 
We may assume without loss of generality that $ADV^\fanr$ never queries
$\fanr$ on the same string more than once, since $\fanr$'s response would be
identical.
%Observe that 
%$ADV^\fanr$ must query $\fanr$ on $g^{uv} \in G$ at some point, since 
%$p_{ADV}(n)$ would be at most $\frac{1}{2}$ if he did not (because $s_b$ 
%would then yield no information about $b$). As in Section~\ref{SEC:OurResult}
%of Chapter~\ref{CH:Fiatshamir}, we will refer to this special $\fanr$ query
%as the ``crucial query''.

We use $ADV^\fanr$ to construct a probabilistic polynomial-time solver $S$
who, given $g^u \in G$ and $g^v \in G$ for randomly chosen $u,v \in
\{1,\ldots,q\}$, outputs $g^{uv} \in G$ with non-negligible probability.
%To prove that DHIES+ is semantically secure in the ROM under the CDH
%assumption, we show how to convert a probabilistic polynomial-time adversary
%$ADV^\fanr$ who correctly outputs $b$ with probability significantly above
%$\frac{1}{2}$ when given a public key $g^{v'}$ and an encryption $e_b =
%(g^{u'}, s_b, t)$ of a randomly chosen bit $b$, where $s_0s_1r =
%\fanr(g^{u'v'})$ and $t = u'\oplus r$, into a probabilistic polynomial-time
%solver $S$ who, given $g^u \in G$ and $g^v \in G$ for randomly chosen $u,v \in
%\{1,\ldots,p-1\}$, outputs $g^{uv} \in G$ with significant probability.
%On input $(g^u, g^v)$, 
$S$ first
%guesses the index of $ADV^\fanr$'s
%``crucial'' $\fanr(g^{uv})$ query by 
randomly chooses $i \in \{1,\ldots,c(n)\}$ and $s,t \in \strs{n}$. He then
simulates $ADV^\fanr$ on $(g^v, (g^u, s, t))$, answering all of $ADV^\fanr$'s
$\fanr$ queries randomly. As soon as $ADV^\fanr$ asks his $i^{th}$ random
oracle query, $\fanr(m)$, $S$ ends the simulation and outputs $m$ as the value
of $g^{uv}$ (if $ADV^\fanr$ terminates before asking the $i^{th}$ query or $m
\notin G$, $S$ outputs some dummy value such as $g^u$). 

Let $A$ be the event that $ADV^\fanr$ succeeds and $B$ be the event that
$ADV^\fanr$ queries $\fanr$ on $g^{uv} \in G$ at some point.
%so that there exists an $i^* \in \{1,\ldots,q_\fanr\}$ such that $ADV^\fanr$'s
%$i^{*^{th}}$ random oracle query is $\fanr(g^{uv})$. 
%is the index of $ADV^\fanr$'s ``crucial'' $\fanr(g^{uv})$ query. 
Observe that
\begin{align*} 
p_{ADV}(n) = \text{Pr}[A] &= \text{Pr}[A,B] +
\text{Pr}[A,\overline{B}]\\ & = \text{Pr}[A,B] + \text{Pr}[A \ | \
\overline{B}]\cdot\text{Pr}[\overline{B}] \\
& \leq \text{Pr}[A,B] + \text{Pr}[A \
| \ \overline{B}],
\end{align*} 
and note that Pr$[A \ | \ \overline{B}] = \frac{1}{2}$, because in that case
$s_b$ yields no information about $b$. Since $ADV^\fanr$ breaks the semantic
security of DHIES+ in the ROM, $p_{ADV}(n) > \frac{1}{2} + \frac{1}{n^d}$
for some $d$ and infinitely many $n$. We therefore have:
%and the choice of $i$ is independent of the simulation of $ADV^\fanr$, 
\begin{align*}
\text{Pr}[A,B] & \geq p_{ADV}(n) - \text{Pr}[A \ | \ \overline{B}]
 = p_{ADV}(n) - \frac{1}{2}\\
& > \frac{1}{2} + \frac{1}{n^d} - \frac{1}{2} = \frac{1}{n^d} \text{ for infinitely many } n.
\end{align*}
Now denote the success probability of $S$ (taken over his random bits, as well
as the choice of $u$ and $v$) by $p_S(n)$, and let $C$ be the event that
$ADV^{\fanr}$'s $i^{th}$ random oracle query is $\fanr(g^{uv})$. Since $i \in
\{1,\ldots,c(n)\}$ is chosen uniformly (and independently of the simulation
of $A^\fanr$), where $c(n) \leq n^c$, we get:
\begin{align*}
p_{S}(n) = \text{Pr}[A, B, C] &= \text{Pr}[C \ | \ A, B] \cdot \text{Pr}[A, B] 
= \frac{1}{c(n)} \cdot \text{Pr}[A, B] \\
&> \frac{1}{c(n)} \cdot \frac{1}{n^d} \geq \frac{1}{n^c}\cdot \frac{1}{n^d}
= \frac{1}{n^{c+d}} \text{ for infinitely many } n.
\end{align*}
This shows that $p_\fans(n)$ is non-negligible, so that  DHIES+ is
semantically secure in the ROM under the CDH assumption. $\blacksquare$
%The lower bound on Pr[$A, B]$ was obtained by observing that 

%$ADV^\fanr$ eventually outputs a bit $b'$ and terminates. $S$ then outputs $a$
%as the value of $g^{uv} \in G$, where $\fanr(a)$ was $ADV^\fanr$'s $i^{th}$
%random oracle query. 
%What is $p_S(n)$, the probability (taken over the random
%bits of $S$, as well as the choice of $u$ and $v$) that $a_i = g^{uv}$?

\subsection{Plaintext awareness}
\label{SEC:Plaintext}
Informally, DHIES+ is plaintext-aware in the ROM because the fact that $u$
(and not merely $g^u$) is incorporated into the ciphertext $e$ enables the
simulator to not only determine if the adversary knows the decryption of $e$,
but to actually decrypt it himself (albeit with negligible error). 
%[some crap about why DHIES+ is plaintext-aware in the ROM.
%Adversary almost certainly won't be able to malleate the encryption he's
%given, $u$ gives away the correct decryption of encryption the adversary
%generated himself].  

More formally, let $ADV^{\fanr,\fand}$ be a probabilistic polynomial-time
adversary who is given a public key $g^v \in G$ and attempts to break the CCA2
security (see Chapter~\ref{CH:Preliminaries}, Section~\ref{SEC:PKEPs}) of
DHIES+ in the ROM. Suppose for simplicity that there is no ``lunchtime
attack'' phase, so that $ADV^{\fanr,\fand}$ gets an encryption $e_b =
(g^u,s_b,u \oplus r)$ of a random bit $b$ (where $s_0s_1r = \fanr(g^{uv})$),
queries the decryption oracle $\fand$ on a bunch of strings
$a_0,a_1,a_2,\ldots$, and finally outputs a bit $b'$. We may assume 
%without loss of generality 
that $a_i \in \strs{3n}$ (so that $a_i = (\alpha_i,
\beta_i, \gamma_i)$ for some $\alpha_i, \beta_i, \gamma_i \in \strs{n}$), 
%and that $a_i \neq e_b$, 
since $\fand$'s reply would definitely be $\perp$ otherwise. 
%$ADV^{\fanr,\fand}$ next queries $\fand$ on some more
%strings $(\alpha_i, \beta_i, \gamma_i)$ and finally outputs a bit $b'$.  
Let $p_{ADV}(n)$ denote the probability that $b' = b$; $p_{ADV}(n)$ taken over
$ADV^{\fanr,\fand}$'s random bits and the randomness of $\fanr$, as well as
the choice of $v$, $u$ and $b$.  We must exhibit a probabilistic
polynomial-time adversary $A^\fanr$, also given $g^v$ and $e_b$, such that
$|p_{ADV}(n)-p_A(n)|$ is negligible; here $p_A(n)$ is the probability that
$A^\fanr$ correctly outputs $b$, taken over his random bits and the randomness
of $\fanr$, as well as the choice of $v$, $u$ and $b$. 

Let $A^\fanr$ simulate $ADV^{\fanr,\fand}$, answering his $\fand(a_i)$ queries
as follows.
%\begin{itemize}
%\item If $\alpha_i \notin G$, answer $\perp$.
$A^\fanr$ checks whether $ADV^{\fanr,\fand}$ has previously queried the random
oracle $\fanr$ on a string $w$ such that $(\alpha_i,\beta_i,\gamma_i) =
(g^{u'},s'_{m},u' \oplus r')$, where $w = g^{u'v}$, $s'_0s'_1r' =
\fanr(w)$, $u' \in \{1,\ldots,q\}$ and $m$ is a bit; he is
basically just trying to determine if $ADV^{\fanr,\fand}$ already knows the
decryption of $a_i$.  $A^\fanr$'s answer is $m$ if such a $w$ exists and $\perp$
otherwise. 
%\end{itemize}

First, observe that whenever $A^\fanr$ answers $\fand(a_i)$ with $m$, so does
$\fand$, since $\fand(g^{u'}, s'_{m}, u' \oplus r') = m$ provided that
$s'_0s'_1r' = \fanr(g^{u'v})$ and $u' \in \{1,\ldots,q\}$. It remains to
show that if $A^\fanr$'s answer is $\perp$, then so is $\fand$'s (almost
certainly, anyway). We consider the following two exhaustive cases.
\begin{itemize}
\item {\bf Case \#1:} $\alpha_i \neq g^u$

If $\alpha_i \notin G$ then $\fand(a_i) =\ \perp$ and we are done. Let us
therefore suppose that $\alpha_i = g^{u'}$ for some $u' \in \{1,\ldots,q\}$,
$u' \neq u$, so that $a_i = (g^{u'},\beta_i,\gamma_i)$. %(note that $\fanr$ was
%queried on $g^{uv}$ in order to generate the encryption $e_b = (g^u, s_b,
%t))$, so 
Set $s'_0s'_1r' = \fanr(g^{u'v})$. Since $G$ has prime order, we are
guaranteed that $g^{u'v} \neq g^{uv}$. This matters because we would otherwise
have $s'_b = s_b$, $r' = r$, and $ADV^{\fanr,\fand}$ has information about
both $s_b$ and $r$ by virtue of having seen $e_b = (g^u,s_b,u\oplus r)$.

\begin{itemize}
\item {\bf Subcase \#1a:} $ADV^{\fanr,\fand}$ has queried $\fanr$ on 
$g^{u'v}$

Since $A^\fanr$'s answer was $\perp$, either $\beta_i \notin \{s'_0, s'_1\}$
or $\gamma_i \neq u' \oplus r'$. In both cases, $\fand(a_i) = \ \perp$.
\item {\bf Subcase \#1b:} $ADV^{\fanr,\fand}$ hasn't queried $\fanr$ on 
$g^{u'v}$

In this case $s_0'$, $s_1'$ and $r'$ are completely random (note that this
assertion is justified only because $g^{u'v} \neq g^{uv}$). The probability
that $a_i$ is a valid encryption, so that $\fand(a_i) \neq \ \perp$, is
therefore 
\[\underbrace{\frac{2}{2^n}}_{s'_0 \text{ or } s'_1} \cdot \: 
\underbrace{\frac{1}{2^n}}_{r'} = \frac{2}{4^n},
\]
which is certainly negligible.
%The reason we care whether $w = g^{uv}$ is that $\fanr$ was definitely
%queried on the latter in order to generate the encryption .
\end{itemize}

\item {\bf Case \#2:} $\alpha_i = g^u$

\begin{itemize}
\item {\bf Subcase \#2a:} $\gamma_i = u \oplus r$

%In this case $a_i = (g^u, \beta_i, u \oplus r)$. 
If $\beta_i \notin \{s_0,s_1\}$ then $\fand(a_i) = \ \perp$ and we are done,
so suppose that $\beta_i = s_{\bar{b}}$ (recall that $ADV^{\fanr,\fand}$ isn't
allowed to query $\fand$ on $e_b = (g^u, s_b, u\oplus r)$).  Since $A^\fanr$'s
answer was $\perp$, we know that $ADV^{\fanr,\fand}$ hasn't queried $\fanr$ on
$g^{uv}$, and consequently has no information about $s_{\bar{b}}$ (because
$ADV^{\fanr,\fand}$ hasn't seen $s_{\bar{b}}$, we may view it as not
having been chosen yet). The probability that $a_i$ is a valid encryption (so
that $\fand(a_i) \neq \ \perp$) is therefore $\frac{1}{2^n}$ in this case,
which is again negligible.  \item {\bf Subcase \#2b:} $\gamma_i \neq u \oplus
r$

In this case $\fand(a_i) = \ \perp$, so we are done.
\end{itemize}

\end{itemize}
%almost certainly is as well.
We can therefore conclude that the difference between $p_A(n)$ and
$p_{ADV}(n)$ is negligible, which means that DHIES+ is plaintext-aware in the
ROM. Since we have already shown in Section~\ref{SEC:Semantic} that DHIES+ is
semantically secure in the ROM under the CDH assumption, this completes our
proof that it is CCA2-secure in the ROM under the CDH assumption.
$\blacksquare$
