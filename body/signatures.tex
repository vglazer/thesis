\chapter{Other Signature Schemes Secure in the ROM}
\label{signatures}

\begin{itemize}

\item \cite{bellare:rompractical} (1993): In their seminal 1993 paper, Bellare
and Rogaway pointed out that the ``hash-then-decrypt'' approach popular among
practitioners yields signature schemes secure in the random oracle model,
provided the undelying trapdoor permutation is one-way. Given a trapdoor
permutation $(f,f^{-1})$ and a random oracle $\fanr$, set the signature
of message $m$ to $\sigma = f^{-1}(\fanr(m))$. To verify that $\sigma$ is
a valid signature of $m$, check whether $f(\sigma) = \fanr(m)$.

\item \cite{bellare:signrsa} (1996): Bellare and Rogaway show how to
probabilistically ``pad'' messages with the aid of a random oracle to obtain a 
signature scheme which is secure in the random oracle model if the RSA
function (or, correspondingly, the Rabin function) is one-way. Since RSA is
typically assumed to be a one-way trapdoor permutation, this result is closely
related to \cite{bellare:rompractical}. However, the new signature scheme
has better ``exact security''. In other words, the reduction between breaking 
the scheme and inverting RSA (or Rabin) is ``tighter'', i.e. more
efficient.

\item \cite{boneh:shortsigs} (2001): Boneh, Lynn and Shacham describe a
signature scheme based on the Tate and Weil pairings which is secure in the random
oracle model provided an analogue of the \emph{computational} Diffie-Hellman
problem is hard for certain elliptic and hyper-elliptic curves. Their scheme is reasonably efficient and
yields signatures twice as short as DSS for a similar level of security.

\end{itemize}
