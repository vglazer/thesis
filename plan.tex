\documentclass[12pt,twoside]{article}

% Standard packages
% standard packages
\usepackage{enumerate,psfrag,amsfonts,amssymb,amsmath,amsthm,url,epic,eepic}
\ifx\pdfoutput\undefined
  \usepackage[dvips]{graphicx}
\else
  \usepackage[pdftex]{graphicx}
\fi


%% Define a new 'leo' style for the package that will use a smaller font.
\makeatletter
\def\url@smallstyle{\@ifundefined{selectfont}{\def\UrlFont{\sf}}{\def\UrlFont{\small\ttfamily}}\Url@do
}
\makeatother

%% Now actually use the newly defined style.
\urlstyle{small}


% Formatting
\input{$THESISDIR/share/format.tex}

% Macros
%*****************
% Important Sets *
%*****************

% The naturals
\newcommand{\nats}{\mathbb{N}}

% The integers
\newcommand{\ints}{\mathbb{Z}}

% The positive positive integers
\newcommand{\pints}{\mathbb{Z^+}}

% The reals
\newcommand{\reals}{\mathbb{R}}

% The interval [0,1]
\newcommand{\zone}{[0,1]}

% Binary strings
\newcommand{\strs}[1]{\{0,1\}^{#1}}


%*****************
% Script Letters *
%*****************

% script B
\newcommand{\fanb}{\mathcal{B}}

% script C
\newcommand{\fanc}{\mathcal{C}}

% script D
\newcommand{\fand}{\mathcal{D}}

% script E
\newcommand{\fane}{\mathcal{E}}

% script F
\newcommand{\fanf}{\mathcal{F}}

% script G
\newcommand{\fang}{\mathcal{G}}

% script H
\newcommand{\fanh}{\mathcal{H}}

% script M
\newcommand{\fanm}{\mathcal{M}}

% script O
\newcommand{\fano}{\mathcal{O}}

% script P
\newcommand{\fanp}{\mathcal{P}}

% script R
\newcommand{\fanr}{\mathcal{R}}

% script S 
\newcommand{\fans}{\mathcal{S}}

% script T 
\newcommand{\fant}{\mathcal{T}}

% script U 
\newcommand{\fanu}{\mathcal{U}}

% script Z
\newcommand{\fanz}{\mathcal{Z}}

%*********************
% Special Constructs *
%*********************

% |x| over n
\newcommand{\overn}[1]{\frac{|#1|}{n}}

% ensembles and circuits
\newcommand{\ens}[2]{\{#1_#2\}_{#2 \in \nats}}

% probability
\newcommand{\pr}[3]{\text{Pr}_{#2 \in #3}[#1]}

% fiat-shamir signature scheme
\newcommand{\fs}[2]{SIG_{#2}(#1)}

% Merkle tree committment
\newcommand{\merk}[2]{TC_{f_#2}(#1)}

% ceiling
\newcommand{\ceil}[1]{\lceil #1 \rceil}

% floor
\newcommand{\floor}[1]{\lfloor #1 \rfloor}

% encoding into strings
\newcommand{\enc}[1]{\langle #1 \rangle}

% function definition
\newcommand{\func}[3]{#1 : #2 \to #3}



% A basic theorem environment
\theoremstyle{plain}
\newtheorem*{thm}{Theorem}

% Author and title
\title{The rise and fall of the Random Oracle Methodology}
\author{Victor Glazer}

\begin{document}

\maketitle
\section*{Timeline}

\begin{itemize}
\item {\bf 1984:} Shafi Goldwasser, Silvio Micali and Ron Rivest describe in
\cite{goldwasser:signatures1} (see \cite{goldwasser:signatures2} for the
journal version) the first signature scheme which is provably secure against 
adaptive chosen message attack under plausible complexity-theoretic 
assumptions. The specific assumption required is that \emph{claw-free} 
trapdoor permutations exist, which is the case if factoring is hard. Note, however, that
claw-free trapdoor permutations may fail to exist even if ordinary
permutations (ones not based on the difficulty of integer factorization, that
is) do. Unfortunately, GMR's ``signature tree'' construction is too
inefficient to be practical.

\item {\bf 1987:} Amos Fiat and Adi Shamir construct a practical signature scheme in \cite{fiat:fsparadigm}
by removing the interaction from a canonical (i.e. three-round, public-coin) identification scheme
using a cryptographic hash function. They show that the resulting signature
scheme is secure against chosen message attack if factoring is hard (which
implies that the underlying identification scheme is secure), 
{\it assuming that the hash function in question is truly random}. This
approach to designing signature schemes later becomes known as the
``Fiat-Shamir paradigm''.

Merkle proposes a tree-like signature scheme based on one-way functions in
\cite{merkle:signatures}, but does not provide a rigorous proof of security.

\item {\bf 1988:} Mihir Bellare and Silvio Micali present a signature scheme
based on trapdoor permutations which is secure if the underlying permutation is
one-way.

\item {\bf 1989:} Moni Naor and Moti Yung show in \cite{naor:signatures} how 
to construct secure signature schemes from weakly collision-resistant hash
functinos, and describe how to obtain such hash functions from one-way
permutations (actually, injective one-way functions, which need not be onto).
Their construction is still ``tree-like'', stateful and inefficient.

\item {\bf 1991:} Charlie Rackoff and Dan Simon present the first 
public-key encryption scheme secure against adaptive chosen ciphertext attack
(i.e. CCA2-secure) in \cite{rackoff:cca2}. Danny Dolev, Cynthia
Dwork and Moni Naor independently develop another CCA2-secure encryption
scheme in \cite{dolev:nonmalleable1} (see \cite{dolev:nonmalleable2} for the
journal version). However, both constructions are based on
\cite{blum:noninterzerok}'s noninteractive zero-knowledge proofs of knowledge 
and therefore aren't practical.

\item {\bf 1992:} Tatsuaki Okamoto develops in \cite{okamoto:discretelogfs} a canonical identification scheme
which is provably secure if discrete log is hard, with a view towards
converting it into a practical signature scheme as per the Fiat-Shamir paradigm.

\item {\bf 1993:} Based on ideas from \cite{fiat:fsparadigm}, Mihir Bellare and Phillip Rogaway propose in 
\cite{bellare:rompractical} a new approach to designing practical public-key
signature and encryption schemes, which later becomes known as the ``Random Oracle
Methodology''. In particular, they exhibit a public-key encryption scheme and
a ``hash-and-sign'' signature scheme which are secure (against
adaptive chosen-ciphertext and chosen-message attacks, respectively) assuming that the
hash functions involved are ``ideal'', i.e. truly random.

%They point out that provably secure protocols (i.e. ones whose breaker
%can be converted into an efficient solver for a problem widely believed to be
%computationally intractable, such as integer factorization or discrete log)
%tend to be prohibitively expensive from a practical standpoint. On the other
%hand, many practical protocols involve the use of hash functions. Standard
%complexity-theoretic assumptions about such functions, namely various flavours of
%collision-resistance, aren't powerful enough to prove these protocols secure.
%However, if one were to model hash functions as random oracles (that is, oracles
%which answer new queries randomly) 

\item {\bf 1994:} Bellare and Rogaway propose a template for practical (i.e.
efficient)
public-key encryption schemes in \cite{bellare:oaep}, called Optimal Assymetric Encryption Padding or
OAEP, which they claim yields PKEPs CCA2-secure in the random
oracle model if instantiated using a one-way trapdoor permutation. OAEP-RSA,
an instantiation of OAEP where RSA is the underlying trapdoor permutation,
is later incorporated into version 2.1 of RSA
Security's PKCS\#1 cryptography standard, as well as IEEE's P1363-2000
public-key cryptography standard.

Cynthia Dwork and Moni Naor present a provably secure practical signature scheme in
\cite{dwork:signatures}.

\item {\bf 1996:} Damg\aa rd and Cramer practical signature
scheme RSA provably secure under RSA assumption (which one?)
\cite{cramer:signatures}.

\item {\bf 1997:} Ran Canetti initiates program to develop hash functions which can
implement random oracles under certain circumstances in
\cite{canetti:hideinfo}. 

Bellare and Rogaway submit a tech report on
OAEP-type constructions to the P1363 standard committee. Charlie thinks their
Diffie-Hellman construction might be wrong.

\item {\bf 1998:} Ran Cenetti, Daniele Micciancio and Omer Reingold refine
some of the definitions of \cite{canetti:hideinfo} in
\cite{canetti:perfecthash}

Ronald Cramer and Victor Shoup develop a practical public-key encryption scheme which
is CCA2 secure (in the standard model) if
the decisional Diffie-Hellman problem is hard in \cite{cramer:cca2secure}.
Their scheme is only slightly less efficient than RSA-OAEP.

Ran Canetti, Oded Goldreich and Shai Halevi exhibit  ``uninstantiable'' public key
encryption and signature schemes in \cite{canetti:romfails}. That is, they
construct public-key encryption and signature schemes which are
secure in the random oracle model, yet insecure in the standard model 
{\it no matter what hash function ensemble is used to implement the random 
oracle}. However, both schemes use Silvio Micali's CS proofs 
(\cite{micali:csproofs}) and are rather contrived and unrealistic. 

\item {\bf 1999:} Cramer and Shoup signature scheme, secure under the ``strong
RSA assumption'' \cite{cramer:signatures2}. A different signature scheme provably secure under the same
assumption independently discovered by Gennaro, Halevi and Rabin
\cite{gennaro:signatures}.

\item {\bf 2001:} 

\item {\bf 2003:} Shafi Goldwasser and Yael Tauman Kalai show in
\cite{goldwasser:fsparadigmfails} that there exist uninstantiable Fiat-Shamir
signature schemes. They exhibit three contrived canonical identification schemes, one of
which must be secure yet yield insecure signature schemes, no matter what
hash function is used to eliminate the verifier's random move. Unfortunately,
their proof has a somewhat nonconstructive flavour and their 
construction, which uses universal arguments (\cite{barak:universalarguments}) and tree committments (\cite{merkle:tree}), 
is highly unrealistic.

\end{itemize}

\section*{In the beginning}
\subsection*{Signature Schemes}
In the late 80s, cryptographers agree that the ``right'' definition of security for
signature schemes is ``security against adaptive chosen-message attack''.
Several ``tree-like'' schemes that provably satisfy this strong
requirement (under standard assumptions) are proposed, but they are
inefficient and hence impractical. On the other hand,
the widely-used ``hash-and-invert'' approach can't be proven secure under
standard assumptions about hash functions (i.e. collision-resistance).

Fiat and Shamir develop a signature scheme (involving a hash function) which is provably secure if
factoring is hard, assuming that the hash function is truly random. Although
hash functions obviously \emph{aren't} truly random (because they have short
descriptions), the scheme is very efficient. 

\subsection*{Public-key Encryption Schemes}
In the early 90s, cryptographers agree that the ``right'' definition of
security for public-key encryption schemes is ``securiy against adaptive
chosen-ciphertext attack'' or, equivalently, ``non-malleability''.
Schemes that provably satisfy this strong requirement (under standard
assumptions) are proposed, but they use ``noninteractive zero knowledge proofs
of knowledge'' and are too inefficient to be practical.

\section*{The Random Oracle Methodology is born}
The state of affairs in the early nineties is as follows: provably secure constructions are inefficient, whereas practical ones lack
proofs of security. Bellare and Rogaway suggest a compromise: schemes
(involving hash functions) which are efficient, yet secure only if the
underlying hash functions 

\section*{The ROM outlives its usefulness}
We now have efficient signature schemes which are provably secure under standard
assumptions (both factoring and discrete log). We also have an efficient public-key encryption
scheme (Cramer-Shoup) which is provably secure if discrete log is hard.
Unfortunately, we \emph{don't} have one
which is secure assuming factoring is hard (i.e. one based on RSA or the Rabin
function). This matters because discrete log may turn out to be easy. We also have RSA-OAEP, which is twice as fast (?) as Cramer-Shoup
and ROM-secure under the standard RSA assumption. 

Micali's computationally sound (CS) proofs are a nice tool widely used in
theoretical cryptography, but they are only known to work in the random oracle
model. Can we ``instantiate'' the oracle somehow, without losing the crucial
``proof-of-knowledge'' property?



% References
\bibliography{$THESISDIR/share/references/random_oracles}

\end{document}

